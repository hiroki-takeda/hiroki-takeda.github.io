%%%%%%%%% JPS abstruct %%%%%%%%%%%%%%%%%%%%%%%%%%%%%%%%%%%%%%%%%%
\documentclass[uplatex, 12pt]{article}
\special{papersize=210truemm,297truemm}

%%%%%%%%% packages %%%%%%%%%%%%%%%%%%%%%%%%%%%%%%%%%%%%%%%%%%%%%%
\usepackage{graphicx} % Include figure files
\usepackage{geometry} %余白設定用package
\geometry{inner=2cm, outer=2cm, top=2cm, bottom=2cm}
\usepackage[dvipdfmx,bookmarks=true,bookmarksnumbered=true,colorlinks=true,linkcolor=blue,citecolor=blue,filecolor=blue,urlcolor=blue]{hyperref} %hyperref packageを使って目次にリンク(しおり)を貼る。
\usepackage[dvipdfmx]{pxjahyper} %日本語文字化け防止
\usepackage[normalem]{ulem} %下線引きたい

\usepackage{multicol} %部分的に二段組
\setlength{\columnsep}{-3.5cm} %二段組の幅調整

\usepackage{otf} % 名前用フォント
\usepackage{gillius} % font inspired by Gill sans 
\usepackage{sectsty} % sectionのスタイル
\allsectionsfont{\gillius}

\newcommand{\comicneue}{\gillius\selectfont}
\DeclareTextFontCommand{\textcomicneue}{\comicneue}

\pagestyle{empty}

% New environment for the long list
\newenvironment{cvlist}{%
	\begin{tabular*}{\textwidth}{@{\extracolsep{\fill}}ll}
}{%
	\end{tabular*}
}

%enumerateで数字に丸印つける。
\newcommand{\ctext}[1]{\ooalign{
\hfil\resizebox{\width}{\height}{#1}\hfil
\crcr
\raise.2mm\hbox{\large$\bigcirc$}}}


%\cvitem{<dates>}{<title>}{<location>}{<description>}
\newcommand{\cvitem}[4]{%
	#1&\parbox[t]{0.83\textwidth}{%
		\normalfont{#2}%
		\hfill%
		{\footnotesize#3}\\%
		#4\vspace{\parsep}%
	}\\
}


%%%%%%%%% header %%%%%%%%%%%%%%%%%%%%%%%%%%%%%%%%%%%%%%%%%%%%%%%%
\begin{document}
\sloppy

\begin{center}
\textcomicneue{\Huge\textbf{Curriculum Vitae}}\\[14pt]

\textcomicneue{\Huge\textbf{ Hiroki Takeda}}\\[5pt]

\end{center}
\date{\today}

\vspace{10pt}
%%%%%%%%%%%%%%%%%%%%%%%%%%%%%%%%%%%%%%%%%%%%%%%%%%%%%%%%%%%

\begin{multicols}{2}
\noindent 
\textcomicneue{\large\textbf{Address:}}\\
\noindent 
Theoretical Astrophysics Group\par
\noindent 
Department of Physics\par
\noindent 
Kyoto University\par
\noindent 
Kyoto 606-8502, Japan\\
\columnbreak

\noindent 
\textcomicneue{\large\textbf{Phone:}} +81-75-753-3876\\
\noindent 
\textcomicneue{\large\textbf{E-mail:} }takeda@tap.scphys.kyoto-u.ac.jp\\
\noindent 
\textcomicneue{\large\textbf{Web site:}} https://hiroki-takeda.github.io/index.html\\
\noindent 
\textcomicneue{\large\textbf{Nationality:}} Japan\\
\noindent 
\textcomicneue{\large\textbf{Birth:}} October 5, 1993\\
\end{multicols}

%%%%%%%%%%%%%%%%%%%%%%%%%%%%%%%%%%%%%%%%%%%%%%%%%%%%%%%%%%%
\section*{Education}

\begin{cvlist} % Environment for a list with descriptions
	%\cvitem{<dates>}{<title>}{<location>}{<description>}
	\cvitem{2018-2021}{Doctor of Philosophy, University of Tokyo,}{}{\emph{Tests of Alternative Theories of Gravity through Gravitational-Wave \\ Polarizations}, \\Completion Date: March, 2021.}\\
	\cvitem{2016-2018}{Master of Science in Physics, University of Tokyo,}{}{\emph{Development of Monolithic Optical System for Lorentz Invariance Test}, \\Completion Date: March, 2018.}\\
	\cvitem{2012-2016}{Bachelor of Science in Physics, Yokohama National University,}{}{\emph{Proposal of Rayleigh Scattering Length Measurement in Liquid Xenon}, \\Completion Date: March, 2016.}
\end{cvlist}

\section*{Employment}
\begin{cvlist} % Environment for a list with descriptions
	%\cvitem{<dates>}{<title>}{<location>}{<description>}
	\cvitem{2023-present}{Program-specific Assistant Professor,}{}{The Hakubi Center for Advanced Research, Kyoto University.}\\
	\cvitem{2023-present}{Collaborative Assistant Professor,}{}{Faculty of Science, Kyoto University.}\\	
	\cvitem{2021-2023}{JSPS Research Fellowship for Young Scientists PD,}{}{Department of Physics, Kyoto University.}\\
	\cvitem{2018-2021}{JSPS Research Fellowship for Young Scientists DC1,}{}{Department of Physics, University of Tokyo.}\\
	\cvitem{2016-2021}{Advanced Leading Graduate Course for Photon Science student,}{}{Department of Physics, University of Tokyo.}
\end{cvlist}

\section*{Awards}
\begin{cvlist} % Environment for a list with descriptions
	%\cvitem{<dates>}{<title>}{<location>}{<description>}
	\cvitem{2021}{Best Student Presentation Award,}{}{Physical Society of Japan.}\\
	\cvitem{2015}{Academic Excellence Award,}{}{Physics and Applied Physics Program, Yokohama National University.}\\
	\cvitem{2015}{Physical Engineering Alumni Association Excellence Award,}{}{Physics and Applied Physics Program, Yokohama National University.}\\
	\cvitem{2014}{Best Poster Award for Investigation Training Program,}{}{Physics and Applied Physics Program, Yokohama National University.}
\end{cvlist}

\section*{Grants}
\begin{cvlist} % Environment for a list with descriptions
	%\cvitem{<dates>}{<title>}{<location>}{<description>}
	\cvitem{2024-2029}{Hakubi project, Kyoto University, Principal Investigator,}{}{Project: \emph{Testing theories of gravity in extreme environments through polarization modes of gravitational waves from compact binary coalescences,}\\ Total budget: 14,320,000 JPY.}\\
	\cvitem{2022-2026}{JSPS Grant-in-Aid for Young Scientists, Principal Investigator,}{}{Project: \emph{Testing theories of gravity on cosmological distance scales by searching for polarization modes in gravitational wave propagation,}\\ Total budget: 3,600,000 JPY.}\\
	\cvitem{2021-2024}{Grant-in-Aid for JSPS Research Fellowship, Principal Investigator,}{}{Project: \emph{Tests of gravity theories in strong gravity field by search for gravitational-wave polarization modes from compact binary mergers,}\\ Total budget: 3,640,000 JPY.}\\
	\cvitem{2018-2021}{Grant-in-Aid for JSPS Research Fellowship, Principal Investigator,}{}{Project: \emph{Tests of Lorentz invariance of photons using an optical ring cavity,}\\ Total budget: 2,800,000 JPY.}
\end{cvlist}

\section*{Teaching}
\begin{cvlist} % Environment for a list with descriptions
	%\cvitem{<dates>}{<title>}{<location>}{<description>}
	\cvitem{2025-}{Undergraduate thesis for 4th year students, Reseach Project P5}{}{Department of Physics, Kyoto University.}\\
	\cvitem{2025-}{Part-time Lecturer, Science of Complex Systems for 3rd- and 4th-year students,}{}{Ryukoku University.}\\
	\cvitem{2022-2023}{Part-time Lecturer, Physics III (Mechanics) for 4th year students,}{}{National Institute of Technology (KOSEN), Maizuru College.}\\
	\cvitem{2018-2019}{Teaching Assistant for Laboratory Class,}{}{Department of Physics, University of Tokyo.}\\
	\cvitem{2012-2018}{Cram School Teacher for Junior High School Students,}{}{Musashi Seminar, Saitama, Japan.}
\end{cvlist}

\section*{Committees}
\noindent
\begin{cvlist} % Environment for a list with descriptions
	%\cvitem{<dates>}{<title>}{<location>}{<description>}
	\cvitem{Jan.~2026}{International Conference "JGRG34", Local Organizing Committee Member.}{}{}
	\cvitem{Oct.~2025}{Workshop "ACG2025", Organizer (Chair).}{}{}
	\cvitem{2025}{Hakubi Center, Division Information Security Technical Officer.}{}{}
	\cvitem{Mar.~2025}{Workshop "Gravitational Waves Related Workshop in Western Japan", Organizer (Chair).}{}{}
	\cvitem{2024}{Hakubi Center, HP Working Group.}{}{}
	\cvitem{2024}{Hakubi Center, PR Group (14th term), Representative.}{}{}
	\cvitem{2020}{KAGRA Scientific Congress, Student Representative.}{}{}
\end{cvlist}

\section*{Memberships}
\begin{cvlist} % Environment for a list with descriptions
	%\cvitem{<dates>}{<title>}{<location>}{<description>}
	\cvitem{2020}{Student Representative, KAGRA Scientific Congress.}{}{}\\
	\cvitem{2019-present}{Member, The Astronomical Society of Japan.}{}{}\\
	\cvitem{2018-present}{Member, LISA Consortium.}{}{}\\
	\cvitem{2017-present}{Member, KAGRA Collaboration.}{}{}\\
	\cvitem{2016-present}{Member, The Physical Society of Japan.}{}{}
\end{cvlist}

\section*{Metrics}
%BEGIN_METRICS%

\noindent
\textbf{Publications:} 2 book(s), 19 first-author paper(s), 66 collaboration paper(s), total 87.\\

\noindent 
\textbf{Presentations:} Invited talks 12. International: oral 12, poster 6.\\
\hspace{0.8cm}Total: 30. \\

\noindent 
\textbf{Outreach:} Lectures 8. Media appearances 3. Writing 3. Interviews 7. Others 2.\\
\hspace{2.75cm}Total: 23. \\

%END_METRICS%

\section*{Publications}
%BEGIN_PUBLICATIONS%
\subsection*{Books}
\begin{enumerate}
\item Hiroki Takeda, “Unveiling the Mysteries of the Vast Universe: Astrophysics from the Age of 14”, KADOKAWA, 2023-03-20, 192 pages.
\item Hiroki Takeda, “Astrophysics Encyclopedia: Understanding the Structure of the Universe at a Glance (Korean Translation)”, Bonus Publishing, 2024-07-10, 200 pages.
\end{enumerate}
\subsection*{Journal Articles}
\begin{enumerate}
\item Tomoya Suzuguchi, Hidetoshi Omiya, \uline{Hiroki Takeda}, “Possibility of Multi-Messenger Observations of Quasi-Periodic Eruptions with X-rays and Gravitational Waves”, arXiv:2505.10488.
\item \uline{Hiroki Takeda}, Takahiro Tanaka, “Quantum decoherence of gravitational waves”, Phys. Rev. D, 111, 104080 (2025). arXiv:2502.18560; DOI: 10.1103/PhysRevD.111.104080
\item Hayato Imafuku, \uline{Hiroki Takeda}, Atsushi Nishizawa, Daiki Watarai, Kipp Cannon, “Statistical biases in parametrized searches for gravitational-wave polarizations”, Phys. Rev. D, 112, 024028 (2025). arXiv:2501.16788; DOI: 10.1103/25ym-x87g
\item \uline{Hiroki Takeda}, Takahiro Tanaka, “Strong lensing of gravitational waves with modified propagation”, Phys. Rev. D, 110, 104050 (2024). arXiv:2404.10809; DOI: 10.1103/PhysRevD.110.104050
\item \uline{Hiroki Takeda}, Shinji Tsujikawa, Atsushi Nishizawa, “Gravitational-wave constraints on scalar-tensor gravity from a neutron star and black-hole binary GW200115”, Phys. Rev. D, 109, 104072 (2024). arXiv:2311.09281; DOI: 10.1103/PhysRevD.109.104072
\item Yusuke Manita, \uline{Hiroki Takeda}, Katsuki Aoki, Tomohiro Fujita, Shinji Mukohyama, “Exploring the spin of ultralight dark matter with gravitational wave detectors”, Phys. Rev. D, 109, 095012 (2024). arXiv:2310.10646; DOI: 10.1103/PhysRevD.109.095012
\item \uline{Hiroki Takeda}, Yusuke Manita, Hidetoshi Omiya, Takahiro Tanaka, “Scalar polarization window in gravitational-wave signals”, PTEP, 2023, 073E01 (2023). arXiv:2304.14430; DOI: 10.1093/ptep/ptad082
\item \uline{Hiroki Takeda}, Soichiro Morisaki, Atsushi Nishizawa, “Search for scalar-tensor mixed polarization modes of gravitational waves”, Phys. Rev. D, 105, 084019 (2022). arXiv:2105.00253; DOI: 10.1103/PhysRevD.105.084019
\item \uline{Hiroki Takeda}, Soichiro Morisaki, Atsushi Nishizawa, “Pure polarization test of GW170814 and GW170817 using waveforms consistent with modified theories of gravity”, Phys. Rev. D, 103, 064037 (2021). arXiv:2010.14538; DOI: 10.1103/PhysRevD.103.064037
\item Koji Nagano, \uline{Hiroki Takeda}, Yuta Michimura, Takashi Uchiyama, Masaki Ando, “Demonstration of a dual-pass differential Fabry-Perot interferometer for future interferometric space gravitational wave antennas”, Class. Quant. Grav., 38, 085018 (2021). arXiv:2008.12462; DOI: 10.1088/1361-6382/abed60
\item Tomotada Akutsu, Fabián Erasmo Peña Arellano, Ayaka Shoda, Yoshinori Fujii, Koki Okutomi, Mark Andrew Barton, Ryutaro Takahashi, Kentaro Komori, Naoki Aritomi, Tomofumi Shimoda, Satoru Takano, \uline{Hiroki Takeda}, Enzo Nicolas Tapia San Martin, Ryohei Kozu, Bungo Ikenoue, Yoshiyuki Obuchi, Mitsuhiro Fukushima, Yoichi Aso, Yuta Michimura, Osamu Miyakawa, Masahiro Kamiizumi, “Compact integrated optical sensors and electromagnetic actuators for vibration isolation systems in the gravitational-wave detector KAGRA”, Rev. Sci. Instrum., 91, 115001 (2020). arXiv:2007.09571; DOI: 10.1063/5.0022242
\item Yuta Michimura, Kentaro Komori, Yutaro Enomoto, Koji Nagano, Atsushi Nishizawa, Eiichi Hirose, Matteo Leonardi, Eleonora Capocasa, Naoki Aritomi, Yuhang Zhao, Raffaele Flaminio, Takafumi Ushiba, Tomohiro Yamada, Li-Wei Wei, \uline{Hiroki Takeda}, Satoshi Tanioka, Masaki Ando, Kazuhiro Yamamoto, Kazuhiro Hayama, Sadakazu Haino, Kentaro Somiya, “Prospects for improving the sensitivity of the cryogenic gravitational wave detector KAGRA”, Phys. Rev. D, 102, 022008 (2020). arXiv:2006.08970; DOI: 10.1103/PhysRevD.102.022008
\item Tomoya Kinugawa, \uline{Hiroki Takeda}, Ataru Tanikawa, Hiroya Yamaguchi, “Probe for Type Ia Supernova Progenitor in Decihertz Gravitational Wave Astronomy”, Astrophys. J., 938, 52 (2022). arXiv:1910.01063; DOI: 10.3847/1538-4357/ac9135
\item \uline{Hiroki Takeda}, Atsushi Nishizawa, Koji Nagano, Yuta Michimura, Kentaro Komori, Masaki Ando, Kazuhiro Hayama, “Prospects for gravitational-wave polarization tests from compact binary mergers with future ground-based detectors”, Phys. Rev. D, 100, 042001 (2019). arXiv:1904.09989; DOI: 10.1103/PhysRevD.100.042001
\item \uline{Hiroki Takeda}, Atsushi Nishizawa, Yuta Michimura, Koji Nagano, Kentaro Komori, Masaki Ando, Kazuhiro Hayama, “Polarization test of gravitational waves from compact binary coalescences”, Phys. Rev. D, 98, 022008 (2018). arXiv:1806.02182; DOI: 10.1103/PhysRevD.98.022008
\item Yuta Michimura, Kentaro Komori, Atsushi Nishizawa, \uline{Hiroki Takeda}, Koji Nagano, Yutaro Enomoto, Kazuhiro Hayama, Kentaro Somiya, Masaki Ando, “Particle swarm optimization of the sensitivity of a cryogenic gravitational wave detector”, Phys. Rev. D, 97, 122003 (2018). arXiv:1804.09894; DOI: 10.1103/PhysRevD.97.122003
\item Kentaro Komori, Yutaro Enomoto, \uline{Hiroki Takeda}, Yuta Michimura, Kentaro Somiya, Masaki Ando, Stefan W. Ballmer, “Direct approach for the fluctuation-dissipation theorem under nonequilibrium steady-state conditions”, Phys. Rev. D, 97, 102001 (2018). arXiv:1803.00585; DOI: 10.1103/PhysRevD.97.102001
\item Yuta Michimura, Tomofumi Shimoda, Takahiro Miyamoto, Ayaka Shoda, Koki Okutomi, Yoshinori Fujii, Hiroki Tanaka, Mark A. Barton, Ryutaro Takahashi, Yoichi Aso, Tomotada Akutsu, Masaki Ando, Yutaro Enomoto, Raffaele Flaminio, Kazuhiro Hayama, Eiichi Hirose, Yuki Inoue, Takaaki Kajita, Masahiro Kamiizumi, Seiji Kawamura, Keiko Kokeyama, Kentaro Komori, Rahul Kumar, Osamu Miyakawa, Koji Nagano, Masayuki Nakano, Naoko Ohishi, Ching Pin Ooi, Fabián Erasmo Peña Arellano, Yoshio Saito, Katsuhiko Shimode, Kentaro Somiya, \uline{Hiroki Takeda}, Takayuki Tomaru, Takashi Uchiyama, Takafumi Ushiba, Kazuhiro Yamamoto, Takaaki Yokozawa, Hirotaka Yuzurihara, “Mirror actuation design for the interferometer control of the KAGRA gravitational wave telescope”, Class. Quant. Grav., 34, 225001 (2017). arXiv:1709.02574; DOI: 10.1088/1361-6382/aa90e3
\item Kiwamu Izumi, Norichika Sago, Tomotada Akutsu, Masaki Ando, Ryuichi Fujita, Kenji Fukunabe, Naoki Kita, Masato Kobayashi, Kentaro Komori, Yuta Michimura, Mitsuru Musha, Koji Nagano, Hiroyuki Nakano, Hiroki Okasaka, Naoki Seto, Ayaka Shoda, Hideyuki Tagoshi, Satoru Takano, \uline{Hiroki Takeda}, Takahiro Tanaka, Kei Yamada, “The current status of contribution activities in Japan for LISA”, Progress of Theoretical and Experimental Physics, 2021, 05A106 (2020). arXiv:https://academic.oup.com/ptep/article-pdf/2021/5/05A106/37953039/ptaa124.pdf; DOI: 10.1093/ptep/ptaa124
\end{enumerate}
\subsection*{Collaboration Papers}
\begin{enumerate}
\item LIGO Scientific, VIRGO, KAGRA Collaboration (including \uline{Hiroki Takeda}), “GWTC-4.0: Population Properties of Merging Compact Binaries”, arXiv:2508.18083.
\item LIGO Scientific, VIRGO, KAGRA Collaboration (including \uline{Hiroki Takeda}), “GWTC-4.0: Updating the Gravitational-Wave Transient Catalog with Observations from the First Part of the Fourth LIGO-Virgo-KAGRA Observing Run”, arXiv:2508.18082.
\item LIGO Scientific, VIRGO, KAGRA Collaboration (including \uline{Hiroki Takeda}), “GWTC-4.0: Methods for Identifying and Characterizing Gravitational-wave Transients”, arXiv:2508.18081.
\item LIGO Scientific, VIRGO, KAGRA Collaboration (including \uline{Hiroki Takeda}), “GWTC-4.0: An Introduction to Version 4.0 of the Gravitational-Wave Transient Catalog”, arXiv:2508.18080.
\item LIGO Scientific, VIRGO, KAGRA Collaboration (including \uline{Hiroki Takeda}), “Open Data from LIGO, Virgo, and KAGRA through the First Part of the Fourth Observing Run”, arXiv:2508.18079.
\item LIGO Scientific, VIRGO, KAGRA Collaboration (including \uline{Hiroki Takeda}), “Upper Limits on the Isotropic Gravitational-Wave Background from the first part of LIGO, Virgo, and KAGRA's fourth Observing Run”, arXiv:2508.20721.
\item LIGO Scientific, VIRGO, KAGRA Collaboration (including \uline{Hiroki Takeda}), “GW231123: a Binary Black Hole Merger with Total Mass 190-265 $M_ødot$”, arXiv:2507.08219.
\item LIGO Scientific, VIRGO, KAGRA Collaboration (including \uline{Hiroki Takeda}), “All-sky search for short gravitational-wave bursts in the first part of the fourth LIGO-Virgo-KAGRA observing run”, arXiv:2507.12374.
\item LIGO Scientific, VIRGO, KAGRA Collaboration (including \uline{Hiroki Takeda}), “All-sky search for long-duration gravitational-wave transients in the first part of the fourth LIGO-Virgo-KAGRA Observing run”, arXiv:2507.12282.
\item LIGO Scientific, VIRGO, KAGRA Collaboration (including \uline{Hiroki Takeda}), “Search for Continuous Gravitational Waves from Known Pulsars in the First Part of the Fourth LIGO-Virgo-KAGRA Observing Run”, Astrophys. J., 983, 99 (2025). arXiv:2501.01495; DOI: 10.3847/1538-4357/adb3a0
\item LIGO Scientific, Virgo, KAGRA Collaboration (including \uline{Hiroki Takeda}), “A Search Using GEO600 for Gravitational Waves Coincident with Fast Radio Bursts from SGR 1935+2154”, Astrophys. J., 977, 255 (2024). arXiv:2410.09151; DOI: 10.3847/1538-4357/ad8de0
\item LIGO Scientific, KAGRA, VIRGO Collaboration (including \uline{Hiroki Takeda}), “Search for Gravitational Waves Emitted from SN 2023ixf”, Astrophys. J., 985, 183 (2025). arXiv:2410.16565; DOI: 10.3847/1538-4357/adc681
\item LIGO Scientific, Virgo,, KAGRA, Swift, Swift-BAT/GUANO Collaboration (including \uline{Hiroki Takeda}), “Swift-BAT GUANO Follow-up of Gravitational-wave Triggers in the Third LIGO–Virgo–KAGRA Observing Run”, Astrophys. J., 980, 207 (2025). arXiv:2407.12867; DOI: 10.3847/1538-4357/ad9749
\item LIGO Scientific, Virgo, KAGRA Collaboration (including \uline{Hiroki Takeda}), “Observation of Gravitational Waves from the Coalescence of a 2.5–4.5 M $_⊙$ Compact Object and a Neutron Star”, Astrophys. J. Lett., 970, L34 (2024). arXiv:2404.04248; DOI: 10.3847/2041-8213/ad5beb
\item LIGO Scientific, Virgo, KAGRA Collaboration (including \uline{Hiroki Takeda}), “Ultralight vector dark matter search using data from the KAGRA O3GK run”, Phys. Rev. D, 110, 042001 (2024). arXiv:2403.03004; DOI: 10.1103/PhysRevD.110.042001
\item LIGO Scientific, Virgo, KAGRA Collaboration (including \uline{Hiroki Takeda}), “Search for Eccentric Black Hole Coalescences during the Third Observing Run of LIGO and Virgo”, Astrophys. J., 973, 132 (2024). arXiv:2308.03822; DOI: 10.3847/1538-4357/ad65ce
\item Fermi Gamma-Ray Burst Monitor Team, LIGO Scientific, Virgo, KAGRA Collaboration (including \uline{Hiroki Takeda}), “A Joint Fermi-GBM and Swift-BAT Analysis of Gravitational-wave Candidates from the Third Gravitational-wave Observing Run”, Astrophys. J., 964, 149 (2024). arXiv:2308.13666; DOI: 10.3847/1538-4357/ad1eed
\item LIGO Scientific, Virgo, KAGRA Collaboration (including \uline{Hiroki Takeda}), “Search for Gravitational-lensing Signatures in the Full Third Observing Run of the LIGO–Virgo Network”, Astrophys. J., 970, 191 (2024). arXiv:2304.08393; DOI: 10.3847/1538-4357/ad3e83
\item LIGO Scientific, Virgo, KAGRA Collaboration (including \uline{Hiroki Takeda}), “Open Data from the Third Observing Run of LIGO, Virgo, KAGRA, and GEO”, Astrophys. J. Suppl., 267, 29 (2023). arXiv:2302.03676; DOI: 10.3847/1538-4365/acdc9f
\item LIGO Scientific, Virgo, KAGRA Collaboration (including \uline{Hiroki Takeda}), “Search for subsolar-mass black hole binaries in the second part of Advanced LIGO's and Advanced Virgo's third observing run”, Mon. Not. Roy. Astron. Soc., 524, 5984--5992 (2023). arXiv:2212.01477; DOI: 10.1093/mnras/stad588
\item LIGO Scientific, Virgo, KAGRA Collaboration (including \uline{Hiroki Takeda}), “Search for Gravitational-wave Transients Associated with Magnetar Bursts in Advanced LIGO and Advanced Virgo Data from the Third Observing Run”, Astrophys. J., 966, 137 (2024). arXiv:2210.10931; DOI: 10.3847/1538-4357/ad27d3
\item LIGO Scientific, Virgo, KAGRA Collaboration (including \uline{Hiroki Takeda}), “Model-based Cross-correlation Search for Gravitational Waves from the Low-mass X-Ray Binary Scorpius X-1 in LIGO O3 Data”, Astrophys. J. Lett., 941, L30 (2022). arXiv:2209.02863; DOI: 10.3847/2041-8213/aca1b0
\item KAGRA Collaboration (including \uline{Hiroki Takeda}), “Noise subtraction from KAGRA O3GK data using Independent Component Analysis”, Class. Quant. Grav., 40, 085015 (2023). arXiv:2206.05785; DOI: 10.1088/1361-6382/acc0cb
\item LIGO Scientific, Virgo, KAGRA Collaboration (including \uline{Hiroki Takeda}), “Search for continuous gravitational wave emission from the Milky~Way center in O3 LIGO-Virgo data”, Phys. Rev. D, 106, 042003 (2022). arXiv:2204.04523; DOI: 10.1103/PhysRevD.106.042003
\item LIGO Scientific, Virgo, KAGRA Collaboration (including \uline{Hiroki Takeda}), “First joint observation by the underground gravitational-wave detector KAGRA with GEO 600”, PTEP, 2022, 063F01 (2022). arXiv:2203.01270; DOI: 10.1093/ptep/ptac073
\item KAGRA Collaboration (including \uline{Hiroki Takeda}), “Performance of the KAGRA detector during the first joint observation with GEO\,600 (O3GK)”, PTEP, 2023, 10A101 (2023). arXiv:2203.07011; DOI: 10.1093/ptep/ptac093
\item LIGO Scientific, Virgo, KAGRA, CHIME/FRB Collaboration (including \uline{Hiroki Takeda}), “Search for Gravitational Waves Associated with Fast Radio Bursts Detected by CHIME/FRB during the LIGO–Virgo Observing Run O3a”, Astrophys. J., 955, 155 (2023). arXiv:2203.12038; DOI: 10.3847/1538-4357/acd770
\item LIGO Scientific, Virgo, KAGRA Collaboration (including \uline{Hiroki Takeda}), “All-sky search for continuous gravitational waves from isolated neutron stars using Advanced LIGO and Advanced Virgo O3 data”, Phys. Rev. D, 106, 102008 (2022). arXiv:2201.00697; DOI: 10.1103/PhysRevD.106.102008
\item LIGO Scientific, Virgo, KAGRA Collaboration (including \uline{Hiroki Takeda}), “All-sky search for continuous gravitational waves from isolated neutron stars using Advanced LIGO and Advanced Virgo O3 data”, Phys. Rev. D, 106, 102008 (2022). arXiv:2201.00697; DOI: 10.1103/PhysRevD.106.102008
\item LIGO Scientific, Virgo, KAGRA Collaboration (including \uline{Hiroki Takeda}), “Search for gravitational waves from Scorpius X-1 with a hidden Markov model in O3 LIGO data”, Phys. Rev. D, 106, 062002 (2022). arXiv:2201.10104; DOI: 10.1103/PhysRevD.106.062002
\item LIGO Scientific, Virgo, KAGRA Collaboration (including \uline{Hiroki Takeda}), “Tests of General Relativity with GWTC-3”, arXiv:2112.06861.
\item LIGO Scientific, Virgo, KAGRA Collaboration (including \uline{Hiroki Takeda}), “Narrowband Searches for Continuous and Long-duration Transient Gravitational Waves from Known Pulsars in the LIGO-Virgo Third Observing Run”, Astrophys. J., 932, 133 (2022). arXiv:2112.10990; DOI: 10.3847/1538-4357/ac6ad0
\item LIGO Scientific, Virgo, KAGRA Collaboration (including \uline{Hiroki Takeda}), “Population of Merging Compact Binaries Inferred Using Gravitational Waves through GWTC-3”, Phys. Rev. X, 13, 011048 (2023). arXiv:2111.03634; DOI: 10.1103/PhysRevX.13.011048
\item LIGO Scientific, Virgo, KAGRA Collaboration (including \uline{Hiroki Takeda}), “Search for Gravitational Waves Associated with Gamma-Ray Bursts Detected by Fermi and Swift during the LIGO–Virgo Run O3b”, Astrophys. J., 928, 186 (2022). arXiv:2111.03608; DOI: 10.3847/1538-4357/ac532b
\item LIGO Scientific, Virgo, KAGRA Collaboration (including \uline{Hiroki Takeda}), “GWTC-3: Compact Binary Coalescences Observed by LIGO and Virgo during the Second Part of the Third Observing Run”, Phys. Rev. X, 13, 041039 (2023). arXiv:2111.03606; DOI: 10.1103/PhysRevX.13.041039
\item LIGO Scientific, Virgo, KAGRA Collaboration (including \uline{Hiroki Takeda}), “Constraints on the Cosmic Expansion History from GWTC–3”, Astrophys. J., 949, 76 (2023). arXiv:2111.03604; DOI: 10.3847/1538-4357/ac74bb
\item LIGO Scientific, Virgo, KAGRA Collaboration (including \uline{Hiroki Takeda}), “Searches for Gravitational Waves from Known Pulsars at Two Harmonics in the Second and Third LIGO-Virgo Observing Runs”, Astrophys. J., 935, 1 (2022). arXiv:2111.13106; DOI: 10.3847/1538-4357/ac6acf
\item LIGO Scientific, Virgo, KAGRA Collaboration (including \uline{Hiroki Takeda}), “All-sky search for gravitational wave emission from scalar boson clouds around spinning black holes in LIGO O3 data”, Phys. Rev. D, 105, 102001 (2022). arXiv:2111.15507; DOI: 10.1103/PhysRevD.105.102001
\item LIGO Scientific, Virgo, KAGRA Collaboration (including \uline{Hiroki Takeda}), “All-sky, all-frequency directional search for persistent gravitational waves from Advanced LIGO's and Advanced Virgo's first three observing runs”, Phys. Rev. D, 105, 122001 (2022). arXiv:2110.09834; DOI: 10.1103/PhysRevD.105.122001
\item LIGO Scientific, Virgo, KAGRA Collaboration (including \uline{Hiroki Takeda}), “Search for continuous gravitational waves from 20 accreting millisecond
x-ray pulsars in O3 LIGO data”, Phys. Rev. D, 105, 022002 (2022). arXiv:2109.09255; DOI: 10.1103/PhysRevD.105.022002
\item LIGO Scientific, Virgo, KAGRA Collaboration (including \uline{Hiroki Takeda}), “Search for Subsolar-Mass Binaries in the First Half of Advanced LIGO's and Advanced Virgo's Third Observing Run”, Phys. Rev. Lett., 129, 061104 (2022). arXiv:2109.12197; DOI: 10.1103/PhysRevLett.129.061104
\item LIGO Scientific, Virgo, KAGRA Collaboration (including \uline{Hiroki Takeda}), “All-sky search for short gravitational-wave bursts in the third Advanced LIGO and Advanced Virgo run”, Phys. Rev. D, 104, 122004 (2021). arXiv:2107.03701; DOI: 10.1103/PhysRevD.104.122004
\item LIGO Scientific, Virgo, KAGRA Collaboration (including \uline{Hiroki Takeda}), “All-sky search for long-duration gravitational-wave bursts in the third Advanced LIGO and Advanced Virgo run”, Phys. Rev. D, 104, 102001 (2021). arXiv:2107.13796; DOI: 10.1103/PhysRevD.104.102001
\item LIGO Scientific, Virgo, KAGRA Collaboration (including \uline{Hiroki Takeda}), “Observation of Gravitational Waves from Two Neutron Star–Black Hole Coalescences”, Astrophys. J. Lett., 915, L5 (2021). arXiv:2106.15163; DOI: 10.3847/2041-8213/ac082e
\item LIGO Scientific, Virgo, KAGRA Collaboration (including \uline{Hiroki Takeda}), “Searches for Continuous Gravitational Waves from Young Supernova Remnants in the Early Third Observing Run of Advanced LIGO and Virgo”, Astrophys. J., 921, 80 (2021). arXiv:2105.11641; DOI: 10.3847/1538-4357/ac17ea
\item LIGO Scientific, Virgo, KAGRA Collaboration (including \uline{Hiroki Takeda}), “Constraints on dark photon dark matter using data from LIGO's and Virgo's third observing run”, Phys. Rev. D, 105, 063030 (2022). arXiv:2105.13085; DOI: 10.1103/PhysRevD.105.063030
\item LIGO Scientific, Virgo, KAGRA Collaboration (including \uline{Hiroki Takeda}), “Search for intermediate-mass black hole binaries in the third observing run of Advanced LIGO and Advanced Virgo”, Astron. Astrophys., 659, A84 (2022). arXiv:2105.15120; DOI: 10.1051/0004-6361/202141452
\item LIGO Scientific, Virgo, KAGRA Collaboration (including \uline{Hiroki Takeda}), “Constraints from LIGO O3 Data on Gravitational-wave Emission Due to R-modes in the Glitching Pulsar PSR J0537–6910”, Astrophys. J., 922, 71 (2021). arXiv:2104.14417; DOI: 10.3847/1538-4357/ac0d52
\item LIGO Scientific, Virgo, KAGRA Collaboration (including \uline{Hiroki Takeda}), “Search for anisotropic gravitational-wave backgrounds using data from Advanced LIGO and Advanced Virgo's first three observing runs”, Phys. Rev. D, 104, 022005 (2021). arXiv:2103.08520; DOI: 10.1103/PhysRevD.104.022005
\item LIGO Scientific, Virgo, KAGRA Collaboration (including \uline{Hiroki Takeda}), “Upper limits on the isotropic gravitational-wave background from Advanced LIGO and Advanced Virgo's third observing run”, Phys. Rev. D, 104, 022004 (2021). arXiv:2101.12130; DOI: 10.1103/PhysRevD.104.022004
\item LIGO Scientific, Virgo, KAGRA Collaboration (including \uline{Hiroki Takeda}), “Constraints on Cosmic Strings Using Data from the Third Advanced LIGO–Virgo Observing Run”, Phys. Rev. Lett., 126, 241102 (2021). arXiv:2101.12248; DOI: 10.1103/PhysRevLett.126.241102
\item LIGO Scientific, Virgo, KAGRA Collaboration (including \uline{Hiroki Takeda}), “Diving below the Spin-down Limit: Constraints on Gravitational Waves from the
Energetic Young Pulsar PSR J0537-6910”, Astrophys. J., 913, L27 (2021). arXiv:2012.12926; DOI: 10.3847/2041-8213/abffcd
\item KAGRA Collaboration (including \uline{Hiroki Takeda}), “Overview of KAGRA: Calibration, detector characterization, physical environmental monitors, and the geophysics interferometer”, PTEP, 2021, 05A102 (2021). arXiv:2009.09305; DOI: 10.1093/ptep/ptab018
\item KAGRA Collaboration (including \uline{Hiroki Takeda}), “Overview of KAGRA : KAGRA science”, arXiv:2008.02921.
\item KAGRA Collaboration (including \uline{Hiroki Takeda}), “Overview of KAGRA: Detector design and construction history”, PTEP, 2021, 05A101 (2021). arXiv:2005.05574; DOI: 10.1093/ptep/ptaa125
\item LISA Collaboration (including \uline{Hiroki Takeda}), “Prospects for Fundamental Physics with LISA”, Gen. Rel. Grav., 52, 81 (2020). arXiv:2001.09793; DOI: 10.1007/s10714-020-02691-1
\item KAGRA Collaboration (including \uline{Hiroki Takeda}), “An arm length stabilization system for KAGRA and future gravitational-wave detectors”, Class. Quant. Grav., 37, 035004 (2020). arXiv:1910.00955; DOI: 10.1088/1361-6382/ab5c95
\item KAGRA Collaboration (including \uline{Hiroki Takeda}), “Application of independent component analysis to the iKAGRA data”, PTEP, 2020, 053F01 (2020). arXiv:1908.03013; DOI: 10.1093/ptep/ptaa056
\item KAGRA Collaboration (including \uline{Hiroki Takeda}), “Vibration isolation system with a compact damping system for power recycling mirrors of KAGRA”, Class. Quant. Grav., 36, 095015 (2019). arXiv:1901.03053; DOI: 10.1088/1361-6382/ab0fcb
\item KAGRA Collaboration (including \uline{Hiroki Takeda}), “First cryogenic test operation of underground km-scale gravitational-wave observatory KAGRA”, Class. Quant. Grav., 36, 165008 (2019). arXiv:1901.03569; DOI: 10.1088/1361-6382/ab28a9
\item KAGRA Collaboration (including \uline{Hiroki Takeda}), “KAGRA: 2.5 Generation Interferometric Gravitational Wave Detector”, Nature Astron., 3, 35--40 (2019). arXiv:1811.08079; DOI: 10.1038/s41550-018-0658-y
\item KAGRA Collaboration (including \uline{Hiroki Takeda}), “Construction of KAGRA: an Underground Gravitational Wave Observatory”, PTEP, 2018, 013F01 (2018). arXiv:1712.00148; DOI: 10.1093/ptep/ptx180
\item LIGO Scientific, Virgo, KAGRA Collaboration (including \uline{Hiroki Takeda}), “Prospects for observing and localizing gravitational-wave transients with Advanced LIGO, Advanced Virgo and KAGRA”, Living Rev. Rel., 19, 1 (2016). arXiv:1304.0670; DOI: 10.1007/s41114-020-00026-9
\item KAGRA Collaboration (including \uline{Hiroki Takeda}), “Vibration isolation systems for the beam splitter and signal recycling mirrors of the KAGRA gravitational wave detector”, Class. Quant. Grav., 38, 065011 (2021). DOI: 10.1088/1361-6382/abd922
\item KAGRA Collaboration (including \uline{Hiroki Takeda}), “The Current Status and Future Prospects of KAGRA, the Large-Scale Cryogenic Gravitational Wave Telescope Built in the Kamioka Underground”, Galaxies, 10, 63 (2022). DOI: 10.3390/galaxies10030063
\item KAGRA Collaboration (including \uline{Hiroki Takeda}), “Overview of KAGRA : Data transfer and management”, PTEP, 2023, 10A102 (2023). DOI: 10.1093/ptep/ptad112
\end{enumerate}

%END_PUBLICATIONS%

\section*{Presentations}
%BEGIN_PRESENTATIONS%
\subsection*{Invited Talks}
\begin{enumerate}
\item \uline{Hiroki Takeda}, “High-Frequency Gravitational Waves and Decoherence”, One day workshop on High Frequency Gravitational waves, KEK, Tsukuba, Japan, August 2025.
\item \uline{Hiroki Takeda}, “Can gravitational waves keep quantum nature?”, Colloquium at Theoretical Quantum Physics, Gravitation and Cosmology at Kyushu University, Fukuoka, Japan, June 2025.
\item \uline{Hiroki Takeda}, “Can gravitational waves retain quantum nature?”, Mini workshop on gravity and cosmology, YITP, Kyoto, Japan, March 2025.
\item \uline{Hiroki Takeda}, “Gravity Meets Quantum Measurements: Testing General Relativity with Gravitational Waves”, 3rd Joint Workshop on General Relativity and Cosmology, Saitama, Japan, March 2025.
\item \uline{Hiroki Takeda}, “Testing Gravity through Gravitational Wave Observations”, Bottom-up particle and gravity fusion theory and cosmological observations, Niigata University, Niigata, Japan, December 2024.
\item \uline{Hiroki Takeda}, “Testing gravity through gravitational-wave observation of compact binary mergers”, Data Oriented Astronomy 2024, The Institute of Statistical Mathematics, Tokyo, Japan, October 2024.
\item \uline{Hiroki Takeda}, “Observations of gravitational waves from compact binary mergers”, The First Young Researchers' Workshop on Multi-Messenger Astronomy, Tokyo, Japan, March 2024.
\item \uline{Hiroki Takeda}, “Probing scalar polarizations in gravitational waves”, GW research exchange meeting, online, June 2023.
\item \uline{Hiroki Takeda}, “Scalar polarization window in gravitational-wave signals from compact binary coalescences”, Seminar, Waseda University, Tokyo, Japan, May 2023.
\item \uline{Hiroki Takeda}, “Polarization tests by DECIGO”, 21st DECIGO workshop, online, December 2022.
\item \uline{Hiroki Takeda}, “Consistent search for polarization modes of gravitational waves from generation to detection”, Observational Cosmology Summer Workshop in 2022, Hamamatsu, Japan, August 2022.
\item \uline{Hiroki Takeda}, “Polarization test of gravitational waves from compact binary coalescences”, Seminar at Institute of Particle Physics and Astrophysics, School of Physics, Huazhong University of Science and Technology, Wuhan, China, November 2018.
\end{enumerate}
\subsection*{International Conferences (Oral)}
\begin{enumerate}
\item \uline{Hiroki Takeda}, Takahiro Tanaka, “Quantum decoherence of gravitational waves”, 24th International Conference on General Relativity 
& 16th Edoardo Amaldi Conference on Gravitational Waves, Glasgow, UK, July 2025.
\item \uline{Hiroki Takeda}, Takahiro Tanaka, “Quantum decoherence of gravitational waves”, 15th annual conference on Relativistic Quantum information (North),
Università degli Studi Federico II, Naples, Italy, June 2025.
\item \uline{Hiroki Takeda}, Shinji Tsujikawa, Atsushi Nishizawa, “Gravitational-wave constraints on scalar-tensor gravity from a black-hole and neutron star binary”, The 32th Workshop on General Relativity and Gravitation in Japan – JGRG32, Nagoya, Japan, November 2023.
\item \uline{Hiroki Takeda}, Yusuke Manita, Hidetoshi Omiya, Takahiro Tanaka, “Scalar polarizations window in gravitational wave signals”, 14th Edoardo Amaldi Conference on Gravitational Waves, online, July 2023.
\item \uline{Hiroki Takeda}, Soichiro Morisaki, Atsushi Nishizawa, “Search for mixture of scalar-tensor polarizations of gravitational waves”, 23rd International Conference on General Relativity and Gravitation, Beijing, China, July 2022.
\item \uline{Hiroki Takeda}, Soichiro Morisaki, Atsushi Nishizawa, “Testing gravity theory with gravitational wave polarizations”, Innovative Area "Gravitational Wave Physics and Astronomy: Genesis" Group A Winter Camp, Kyoto, Japan, January 2021.
\item \uline{Hiroki Takeda}, Soichiro Morisaki, Atsushi Nishizawa, “Scalar-tensor mixed polarization search for gravitational waves from compact binary coalescences”, Gravitational Wave Physics and Astronomy Workshop 2021, Hannover, Germany, December 2021.
\item \uline{Hiroki Takeda}, Atsushi Nishizawa, Soichiro Morisaki, “Polarization tests of GW170814 and GW170817 using waveforms consistent with alternative theories of gravity”, 7th KAGRA International Workshop, online (Taiwan in-person), December 2020.
\item \uline{Hiroki Takeda}, Tomoya Kinugawa, Hiroya Yamaguchi, “Ability of DECIGO to constrain the Type Ia supernova progenitor system”, Gravitational Wave Physics and Astronomy Workshop 2019, Tokyo, Japan, October 2019.
\item \uline{Hiroki Takeda}, Atsushi Nishizawa, Yuta Michimura, Koji Nagano, Kentaro Komori, Masaki Ando, Kazuhiro Hayama, “Probing nontensorial polarization of inspiral gravitational waves with the third-generation detectors”, JGRG28, Tokyo, Japan, November 2018.
\item \uline{Hiroki Takeda}, Atsushi Nishizawa, Yuta Michimura, Koji Nagano, Kentaro Komori, Masaki Ando, Kazuhiro Hayama, “Polarization test of gravitational waves from compact binary coalescences”, 15th Marcel Grossmann Meeting, Rome, Italy, July 2018.
\item \uline{Hiroki Takeda}, Atsushi Nishizawa, Yuta Michimura, Koji Nagano, Kentaro Komori, Masaki Ando, Kazuhiro Hayama, “Parameter estimation with inspiral waveforms of compact binary coalescences including nontensorial gravitational waves polarizations”, 19th KAGRA face-to-face meeting, Osaka, Japan, May 2018.
\end{enumerate}
\subsection*{International Conferences (Poster)}
\begin{enumerate}
\item \uline{Hiroki Takeda}, Takahiro Tanaka, “Strong lensing of gravitational waves with modified propagation”, COSMO'24, Kyoto, Japan, October 2024.
\item \uline{Hiroki Takeda}, Takahiro Tanaka, “Strong gravitational lensing of gravitational waves with modified propagation”, LIGO-Virgo-KAGRA meeting, Barcelona, Spain, September 2024.
\item \uline{Hiroki Takeda}, Yusuke Manita, Hidetoshi Omiya, Takahiro Tanaka, “Scalar gravitational wave and fifth force”, 28th KAGRA Face to Face Meeting, online, December 2022.
\item \uline{Hiroki Takeda}, Atsushi Nishizawa, Soichiro Morisaki, “Search for scalar-tensor mixed polarization of gravitational waves”, 14th Edoardo Amaldi Conference on Gravitational Waves, online, July 2021.
\item \uline{Hiroki Takeda}, Atsushi Nishizawa, Soichiro Morisaki, “Tests of alternative theories of gravity through gravitational-wave polarization modes”, 26th Face to Face Meeting, online, December 2020.
\item \uline{Hiroki Takeda}, Atsushi Nishizawa, Yuta Michimura, Koji Nagano, Kentaro Komori, Masaki Ando, Kazuhiro Hayama, “Prospects for gravitational-wave polarization test from compact binary coalescences with next-generation detectors”, GR22 + Amaldi13, Valencia, Spain, July 2019.
\end{enumerate}

%END_PRESENTATIONS%

\section*{Outreach}
%BEGIN_ACTIVITIES%
\subsection*{Lectures}
\begin{enumerate}
\item “The Universe: What We Still Don't Understand”, Research Exchange Meeting of Seven Kanto SSH-Designated Girls' High Schools, Lecture, July 2025.
\item “What I thought and what I didn’t — The value of inefficiency starting from university entrance”, Yokohama National University Doctoral Student Support, Lecture, February 2025.
\item “Beyond gender boundaries before the mysteries of the universe — Thinking about STEM career paths for women with a young physicist”, Flat Citizen Seminar, Lecture, July 2024.
\item “Probing the depths of the universe with ripples in spacetime — From gravitational wave generation to cutting-edge detection technology”, Recurrent Learning Program, Lecture, December 2023.
\item “Probing the depths of the universe with ripples in spacetime”, Shibaura Institute of Technology Kashiwa Junior and Senior High School, Lecture, December 2023.
\item “Testing General Relativity and Gravitational Wave Observation”, KagaQ Tsukiyo Science, Lecture, November 2023.
\item “Exploring the Mysteries of the Universe with Gravity”, Kyoto University Faculty of Science Orientation for High School Girls, Lecture, March 2023.
\item “Probing the Universe with Gravitational Waves”, Minoh Jiyu Gakuen High School, Lecture, August 2022.
\end{enumerate}
\subsection*{Media Appearances}
\begin{enumerate}
\item “Researchers’ Twitter Strategy”, Kyoto University Hakubi Center Symposium, Guest Speaker, November 2023.
\item “Ashita no College”, TBS Radio, Guest Live Appearance, February 2023.
\item “Kyoto One Day”, NHK Kyoto, Guest Live Appearance, July 2022.
\end{enumerate}
\subsection*{Writing}
\begin{enumerate}
\item ““Mercury’s sky is pitch black” - An astrophysicist explains “Why the sky is blue in a way even elementary schoolers can understand””, President Online, Article, February 2025.
\item “The hidden financial realities of the popular children's career “Researcher””, President Online, Article, April 2024.
\item “Disintegrated in less than 0.000001 seconds... A physicist answers a frequently asked question: “What happens if a person falls into a black hole?””, President Online, Article, December 2023.
\end{enumerate}
\subsection*{Interviews}
\begin{enumerate}
\item “Jo-Ha-Kyu”, Asahi Shimbun Morning Edition, Interview Article, October 2023.
\item “Good Gourmet”, Asahi Shimbun Evening Edition, Interview Article, October 2023.
\item “Mizu Setsu”, Mainichi Shimbun Morning Edition, Interview Article, October 2023.
\item “President Family Autumn Issue”, President Inc., Interview Article, September 2023.
\item “47NEWS”, Kyodo News Agency, Interview Article, May 2023.
\item “The Gate of News”, Yomiuri Shimbun, Interview Article, April 2023.
\item “Yuraku Note”, Mainichi Shimbun, Interview Article, June 2022.
\end{enumerate}
\subsection*{Others}
\begin{enumerate}
\item “Doctor Idol Project: Final Audition”, Akihabara UDX Event, Concept & Planning, October 2023.
\item “4-Panel Universe”, YouTube Channel, Total views: 278,000, February 2022.
\end{enumerate}

%END_ACTIVITIES%


\end{document}