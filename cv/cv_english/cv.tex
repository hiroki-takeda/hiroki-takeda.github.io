%%%%%%%%% JPS abstruct %%%%%%%%%%%%%%%%%%%%%%%%%%%%%%%%%%%%%%%%%%
\documentclass[uplatex, 12pt]{article}
\special{papersize=210truemm,297truemm}

%%%%%%%%% packages %%%%%%%%%%%%%%%%%%%%%%%%%%%%%%%%%%%%%%%%%%%%%%
\usepackage{graphicx} % Include figure files
\usepackage{geometry} %余白設定用package
\geometry{inner=2cm, outer=2cm, top=2cm, bottom=2cm}
\usepackage[dvipdfmx,bookmarks=true,bookmarksnumbered=true,colorlinks=true,linkcolor=blue,citecolor=blue,filecolor=blue,urlcolor=blue]{hyperref} %hyperref packageを使って目次にリンク(しおり)を貼る。
\usepackage[dvipdfmx]{pxjahyper} %日本語文字化け防止
\usepackage[normalem]{ulem} %下線引きたい

\usepackage{multicol} %部分的に二段組
\setlength{\columnsep}{-3.5cm} %二段組の幅調整

\usepackage{otf} % 名前用フォント
\usepackage{gillius} % font inspired by Gill sans 
\usepackage{sectsty} % sectionのスタイル
\allsectionsfont{\gillius}

\newcommand{\comicneue}{\gillius\selectfont}
\DeclareTextFontCommand{\textcomicneue}{\comicneue}

\pagestyle{empty}

% New environment for the long list
\newenvironment{cvlist}{%
	\begin{tabular*}{\textwidth}{@{\extracolsep{\fill}}ll}
}{%
	\end{tabular*}
}

%enumerateで数字に丸印つける。
\newcommand{\ctext}[1]{\ooalign{
\hfil\resizebox{\width}{\height}{#1}\hfil
\crcr
\raise.2mm\hbox{\large$\bigcirc$}}}


%\cvitem{<dates>}{<title>}{<location>}{<description>}
\newcommand{\cvitem}[4]{%
	#1&\parbox[t]{0.83\textwidth}{%
		\normalfont{#2}%
		\hfill%
		{\footnotesize#3}\\%
		#4\vspace{\parsep}%
	}\\
}


%%%%%%%%% header %%%%%%%%%%%%%%%%%%%%%%%%%%%%%%%%%%%%%%%%%%%%%%%%
\begin{document}

\begin{center}
\textcomicneue{\Huge\textbf{Curriculum Vitae}}\\[14pt]

\textcomicneue{\Huge\textbf{ Hiroki Takeda}}\\[5pt]

\end{center}
\date{\today}

\vspace{10pt}
%%%%%%%%%%%%%%%%%%%%%%%%%%%%%%%%%%%%%%%%%%%%%%%%%%%%%%%%%%%

\begin{multicols}{2}
\noindent 
\textcomicneue{\large\textbf{Address:}}\\
\noindent 
Theoretical Astrophysics Group\par
\noindent 
Department of Physics\par
\noindent 
Kyoto University\par
\noindent 
Kyoto 606-8502, Japan\\
\columnbreak

\noindent 
\textcomicneue{\large\textbf{Phone:}} +81-75-753-3876\\
\noindent 
\textcomicneue{\large\textbf{E-mail:} }takeda@tap.scphys.kyoto-u.ac.jp\\
\noindent 
\textcomicneue{\large\textbf{Web site:}} https://hiroki-takeda.github.io/index.html\\
\noindent 
\textcomicneue{\large\textbf{Nationality:}} Japan\\
\noindent 
\textcomicneue{\large\textbf{Birth:}} October 5, 1993\\
\end{multicols}

%%%%%%%%%%%%%%%%%%%%%%%%%%%%%%%%%%%%%%%%%%%%%%%%%%%%%%%%%%%
\section*{Education}

\begin{cvlist} % Environment for a list with descriptions
	%\cvitem{<dates>}{<title>}{<location>}{<description>}
	\cvitem{2018-2021}{Doctor of Philosophy, University of Tokyo,}{}{\emph{Tests of Alternative Theories of Gravity through Gravitational-Wave \\ Polarizations}, \\Completion Date: March, 2021.}\\
	\cvitem{2016-2018}{Master of Science in Physics, University of Tokyo,}{}{\emph{Development of Monolithic Optical System for Lorentz Invariance Test}, \\Completion Date: March, 2018.}\\
	\cvitem{2012-2016}{Bachelor of Science in Physics, Yokohama National University,}{}{\emph{Proposal of Rayleigh Scattering Length Measurement in Liquid Xenon}, \\Completion Date: March, 2016.}
\end{cvlist}

\section*{Employment}
\begin{cvlist} % Environment for a list with descriptions
	%\cvitem{<dates>}{<title>}{<location>}{<description>}
	\cvitem{2021-present}{JSPS Research Fellowship for Young Scientists PD,}{}{Department of Physics, Kyoto University.}\\
	\cvitem{2018-2021}{JSPS Research Fellowship for Young Scientists DC1,}{}{Department of Physics, University of Tokyo.}\\
	\cvitem{2016-2021}{Advanced Leading Graduate Course for Photon Science student,}{}{Department of Physics, University of Tokyo.}
\end{cvlist}

\section*{Awards}
\begin{cvlist} % Environment for a list with descriptions
	%\cvitem{<dates>}{<title>}{<location>}{<description>}
	\cvitem{2021}{Best Student Presentation Award,}{}{Physical Society of Japan.}\\
	\cvitem{2015}{Academic Excellence Award,}{}{Physics and Applied Physics Program, Yokohama National University.}\\
	\cvitem{2015}{Physical Engineering Alumni Association Excellence Award,}{}{Physics and Applied Physics Program, Yokohama National University.}\\
	\cvitem{2014}{Best Poster Award for Investigation Training Program,}{}{Physics and Applied Physics Program, Yokohama National University.}
\end{cvlist}

\section*{Grants}
\begin{cvlist} % Environment for a list with descriptions
	%\cvitem{<dates>}{<title>}{<location>}{<description>}
\cvitem{2022-2026}{JSPS Grant-in-Aid for Young Scientists, Principal Investigator,}{}{Project: \emph{Testing theories of gravity on cosmological distance scales by searching for polarization modes in gravitational wave propagation,}\\ Total budget: 3,600,000 JPY.}\\
	\cvitem{2021-2024}{Grant-in-Aid for JSPS Research Fellowship, Principal Investigator,}{}{Project: \emph{Tests of gravity theories in strong gravity field by search for gravitational-wave polarization modes from compact binary mergers,}\\ Total budget: 3,640,000 JPY.}\\
	\cvitem{2018-2021}{Grant-in-Aid for JSPS Research Fellowship, Principal Investigator,}{}{Project: \emph{Tests of Lorentz invariance of photons using an optical ring cavity,}\\ Total budget: 2,800,000 JPY.}
\end{cvlist}

\section*{Teaching}
\begin{cvlist} % Environment for a list with descriptions
	%\cvitem{<dates>}{<title>}{<location>}{<description>}
	\cvitem{2022-2023}{Part-time Lecturer, Physics III (Mechanics) for 4th year students,}{}{National Institute of Technology (KOSEN), Maizuru College.}\\
	\cvitem{2018-2019}{Teaching Assistant for Laboratory Class,}{}{Department of Physics, University of Tokyo.}\\
	\cvitem{2012-2018}{Cram School Teacher for Junior High School Students,}{}{Musashi Seminar, Saitama, Japan.}
\end{cvlist}

\section*{Memberships}
\begin{cvlist} % Environment for a list with descriptions
	%\cvitem{<dates>}{<title>}{<location>}{<description>}
	\cvitem{2020}{Student Representative, KAGRA Scientific Congress.}{}{}\\
	\cvitem{2019-present}{Member, The Astronomical Society of Japan.}{}{}\\
	\cvitem{2018-present}{Member, LISA Consortium.}{}{}\\
	\cvitem{2017-present}{Member, KAGRA Collaboration.}{}{}\\
	\cvitem{2016-present}{Member, The Physical Society of Japan.}{}{}
\end{cvlist}

\section*{Metrics}
\noindent 
\textbf{Publications:} 1 Book, 6 First-author Refereed Papers (104 citations), 9 Other Refereed Papers, 47 Collaboration Papers, Total: 63 Publications (3964 citations, h-index21, g-index56).\\

\noindent 
\textbf{Presentations:}  5 Invited Talks. International Conference: 10 oral presentations, 4 poster presentations. Domestic Conference: 14 oral presentations, 1 poster presentation. Total: 34 Presentations.\\

\noindent 
\textbf{Outreach:}  4 Talks. 8 Interviews.  2 Media. Others 1. Total: 15. \\

\section*{Publications}
*Refereed papers are underlined, non-refereed papers are not underlined, and major papers are circled in serial numbers.

\subsection*{Books}
\begin{enumerate}
\item[{1}.] \uline{Hiroki Takeda}, "\emph{-Unveiling the mysteries of the too large universe- Astrophysics from the age of 14}", KADOKAWA, Mar. 2023, 192 pages.
\end{enumerate}

\noindent

\subsection*{Papers}
\begin{enumerate}
\setcounter{enumi}{1}
\item \uline{Hiroki Takeda}, Shinji Tsujikawa, and Atsushi Nishizawa,
"\emph{Gravitational-wave constraints on scalar-tensor gravity from a neutron star and black-hole binary GW200115}",
Physical Review D, American Physical Society, under review (2023).
(arXiv:22311.09281)\\

   \item Yusuke Manita, \uline{Hiroki Takeda}, Katsuki Aoki, Tomohiro Fujita, and Shinji Mukohyama
"\emph{Exploring spin of ultralight dark matter with gravitational wave detectors}",
Physical Review D, American Physical Society, under review (2023).
(arXiv:2310.10646)\\

\item[\uline{\ctext{4}}.] \uline{Hiroki Takeda}, Yusuke Manita, Hidetoshi Omiya and Takahiro Tanaka, "\emph{Scalar polarization window in gravitational-wave signals}", Progress of Theoretical and Experimental Physics, Physical Society of Japan, {\bf 2023}, 073E01 (2023). (arXiv:2304.14430)\\

\item[\uline{5}.] Tomoya Kinugawa, \uline{Hiroki Takeda}, Ataru Tanikawa and Hiroya Yamaguchi, "\emph{Probe for Type Ia supernova progenitor in decihertz gravitational wave astronomy}", Astrophysical Journal, {\bf 938}, 52 (2022). (arXiv:1910.01063)\\

\item[\uline{\ctext{6}}.] \uline{Hiroki Takeda}, Soichiro Morisaki, and Atsushi Nishizawa,
"\emph{Search for scalar-tensor mixed polarization modes of gravitational waves}",
Physical Review D, American Physical Society, {\bf 105}, 084019 (2022).
(arXiv:2105.00253)\\

   \item[\uline{7}.] \uline{Hiroki Takeda}, Soichiro Morisaki, and Atsushi Nishizawa, "\emph{Pure polarization test of GW170814 and GW170817 using waveforms consistent with modified theories of gravity}", Physical Review D, American Physical Society, {\bf 103}, 064037 (2021). (arXiv:2010.14538)\\
   
\item[\uline{8}.] Koji Nagano, \uline{Hiroki Takeda}, Yuta Michimura, Takashi Uchiyama, and Masaki Ando, "\emph{Demonstration of a dual-pass differential Fabry-Perot interferometer for future interferometric space gravitational wave antennas}", Classical and Quantum Gravity, IOP Publishing, {\bf 38}, 085018 (2021). (arXiv:2008.12462)\\
  
\item[\uline{9}.] Tomotada Akutsu, Fabi\UTF{00E1}n Erasmo Pe\UTF{00F1}a Arellano, Ayaka Shoda, Yoshinori Fujii, Koki Okutomi, Mark Andrew Barton, Ryutaro Takahashi, Kentaro Komori, Naoki Aritomi, Tomofumi Shimoda, Satoru Takano, \uline{Hiroki Takeda}, Enzo Nicolas Tapia San Martin, Ryohei Kozu, Bungo Ikenoue, Yoshiyuki Obuchi, Mitsuhiro Fukushima, Yoichi Aso, Yuta Michimura, Osamu Miyakawa, and Masahiro Kamiizumi, "\emph{Compact integrated optical sensors and electromagnetic actuators for vibration isolation systems in the gravitational-wave detector KAGRA}", Review of Scientific Instruments, American Institute of Physics, {\bf 91}, 115001 (2020). (arXiv:2007.09571)\\

\item[\uline{10}.] Yuta Michimura, Kentaro Komori, Yutaro Enomoto, Koji Nagano, Atsushi Nishizawa, Eiichi Hirose, Matteo Leonardi, Eleonora Capocasa, Naoki Aritomi, Yuhang Zhao, Raffaele Flaminio, Takafumi Ushiba, Tomohiro Yamada, Li-Wei Wei, \uline{Hiroki Takeda}, Satoshi Tanioka, Masaki Ando, Kazuhiro Yamamoto, Kazuhiro Hayama, Sadakazu Haino, and Kentaro Somiya "\emph{Prospects for improving the sensitivity of the cryogenic gravitational wave detector KAGRA}", Physical Review D , American Physical Society, {\bf 102}, 022008 (2020). (arXiv:2006.08970)\\

\item[\uline{11}.] Kiwamu Izumi, Norichika Sago, Tomotada Akutsu, Masaki Ando, Ryuichi Fujita, Kenji Fukunabe, Naoki Kita, Masato Kobayashi, Kentaro Komori, Yuta Michimura, Mitsuru Musha, Koji Nagano, Hiroyuki Nakano, Hiroki Okasaka, Naoki Seto, Ayaka Shoda, Hideyuki Tagoshi, Satoru Takano, \uline{Hiroki Takeda}, Takahiro Tanaka, and Kei Yamada, "\emph{The current status of contribution activities in Japan for LISA}", Progress of Theoretical and Experimental Physics, Physical Society of Japan, {\bf 2020}, ptaa124 (2020).\\

\item[\uline{\ctext{12}}.] \uline{Hiroki Takeda}, Atsushi Nishizawa, Yuta Michimura, Koji Nagano, Kentaro Komori, Masaki Ando, and Kazuhiro Hayama, "\emph{Prospects for gravitational-wave polarization tests from compact binary mergers with future ground-based detectors}", Physical Review D , American Physical Society, {\bf 100}, 042001 (2019). (arXiv:1904.09989)\\

\item[\uline{\ctext{13}}.] \uline{Hiroki Takeda}, Atsushi Nishizawa, Yuta Michimura, Koji Nagano, Kentaro Komori, Masaki Ando, and Kazuhiro Hayama, " \emph{Polarization test of gravitational waves from compact binary coalescences}",Physical Review D , American Physical Society, {\bf 98}, 022008 (2018). (arXiv:1806.02182)\\

\item[\uline{14}.] Yuta Michimura, Kentaro Komori, Atsushi Nishizawa, \uline{Hiroki Takeda}, Koji Nagano, Yutaro Enomoto, Kazuhiro Hayama, Kentaro Somiya, and Masaki Ando, "\emph{Particle swarm optimization of the sensitivity of cryogenic gravitational wave detector}", Physical Review D , American Physical Society, {\bf 97}, 122003, (2018). (arXiv:1804.09894)\\

\item[\uline{15}.] Kentaro Komori, Yutaro Enomoto, \uline{Hiroki Takeda}, Yuta Michimura, Kentaro Somiya, Masaki Ando, and Stefan W. Ballmer, "\emph{Direct Approach for the Fluctuation-Dissipation Theorem under Non-Equilibrium Steady-State Conditions}", Physical Review D , American Physical Society, {\bf 97}, 102001 (2018). (arXiv:1803.00585)\\

\item[\uline{16}.] Yuta Michimura, Tomofumi Shimoda, Takahiro Miyamoto, Ayaka Shoda, Koki Okutomi, Yoshinori Fujii, \uline{Hiroki Tanaka}, Mark A. Barton, Ryutaro Takahashi, Yoichi Aso, Tomotada Akutsu, Masaki Ando, Yutaro Enomoto, Raffaele Flaminio, Kazuhiro Hayama, Eiichi Hirose, Yuki Inoue, Takaaki Kajita, Masahiro Kamiizumi, Seiji Kawamura, Keiko Kokeyama, Kentaro Komori, Rahul Kumar, Osamu Miyakawa, Koji Nagano, Masayuki Nakano, Naoko Ohishi, Ching Pin Ooi, Fabi\UTF{00E1}n Erasmo Pe\UTF{00F1}a Arellano, Yoshio Saito, Katsuhiko Shimode, Kentaro Somiya, Hiroki Takeda, Takayuki Tomaru, Takashi Uchiyama, Takafumi Ushiba, Kazuhiro Yamamoto, Takaaki Yokozawa, and Hirotaka Yuzurihara, "\emph{Mirror actuation design for the interferometer control of the KAGRA gravitational wave telescope}", Classical and Quantum Gravity, IOP Publishing, {\bf 34}, 225001 (2017). (arXiv:1709.02574)
\end{enumerate}

\subsection*{Collaboration papers \\(LIGO-Virgo-KAGRA, KAGRA, LISA, DECIGO)}
\begin{enumerate}
\setcounter{enumi}{16}
\item The LIGO Scientific Collaboration, the Virgo Collaboration, and the KAGRA Collaboration (R. Abbott, ..., \uline{Hiroki Takeda}, ..., et al.), "\emph{Search for gravitational-lensing signatures in the full third observing run of the LIGO-Virgo network}", (2023). (arXiv:2304.08393)\\

\item The LIGO Scientific Collaboration, the Virgo Collaboration, and the KAGRA Collaboration (R. Abbott, ..., \uline{Hiroki Takeda}, ..., et al.), "\emph{Open data from the third observing run of LIGO, Virgo, KAGRA and GEO}", (2023). (arXiv:2302.03676)\\

\item The LIGO Scientific Collaboration, the Virgo Collaboration, and the KAGRA Collaboration (R. Abbott, ..., \uline{Hiroki Takeda}, ..., et al.) plus S. Shandera and D. Jeong, "\emph{Search for subsolar-mass black hole binaries in the second part of Advanced LIGO and Virgo's third observing run}", (2022). (arXiv:2212.01477)\\

\item The LIGO Scientific Collaboration, the Virgo Collaboration, the KAGRA Collaboration (R. Abbott, ..., \uline{Hiroki Takeda}, ..., et al.), "\emph{Search for gravitational-wave transients associated with magnetar bursts in Advanced LIGO and Advanced Virgo data from the third observing run}", The Astrophysical Journal, IOP Publishing, (2022). (arXiv:2210.10931)\\

\item[\uline{21}.] The LIGO Scientific Collaboration, the Virgo Collaboration, the KAGRA Collaboration (R. Abbott, ..., \uline{Hiroki Takeda}, ..., et al.), "\emph{Model-based cross-correlation search for gravitational waves from the low-mass X-ray binary Scorpius X-1 in LIGO O3 data }", The Astrophysical Journal Letters, IOP Publishing, {\bf 941}, L30 (2022). (arXiv:2209.02863)\\

\setcounter{enumi}{21}
\item The KAGRA Collaboration (H. Abe, ..., \uline{Hiroki Takeda}, ..., et al.), "\emph{Noise subtraction from KAGRA O3GK data using Independent Component Analysis}", (2022). (arXiv:2206.05785)\\

\item[\uline{23}.] The KAGRA Collaboration (H. Abe, ..., \uline{Hiroki Takeda}, ..., et al.), "\emph{The current status and future prospects of KAGRA, the large-scale cryogenic gravitational wave telescope built in the Kamioka underground}", Galaxies, MDPI, {\bf 10}, 63 (2022). \\

\item[\uline{24}.] The LIGO Scientific Collaboration, the Virgo Collaboration, and the KAGRA Collaboration (R. Abbott, ..., \uline{Hiroki Takeda}, ..., et al.),
"Search for continuous gravitational wave emission from the Milky Way center in O3 LIGO-Virgo data", Physical Review D, American Physical Society, {\bf 106}, 042003 (2022).
(arXiv:2204.04523)\\

\setcounter{enumi}{24}
\item The LIGO Scientific Collaboration, the Virgo Collaboration, and the KAGRA Collaboration (R. Abbott, ..., \uline{Hiroki Takeda}, ..., et al.),
"\emph{Search for gravitational waves associated with Fast Radio Bursts Detected by CHIME/FRB During the LIGO-Virgo Observing Run O3a}",  The Astrophysical Journal, IOP Publishing,  (2022). (arXiv:2203.12038)\\

\item[\uline{26}.] The LIGO Scientific Collaboration, the Virgo Collaboration, and the KAGRA Collaboration (R. Abbott, ..., \uline{Hiroki Takeda}, ..., et al.), "\emph{First joint observation by the underground gravitational-wave detector KAGRA with GEO 600}", Progress of Theoretical and Experimental Physics, Physical Society of Japan, {\bf 2022}, 063F01 (2022). (arXiv:2203.01270)\\

\item[\uline{27}.] The LIGO Scientific Collaboration, the Virgo Collaboration, and the KAGRA Collaboration (R. Abbott, ..., \uline{Hiroki Takeda}, ..., et al.), "\emph{Search for gravitational waves from Scorpius X-1 with a hidden Markov model in O3 LIGO data}", Physical Review D, American Physical Society, {\bf 106} 062002(2022). (arXiv:2201.10104)\\

\item[\uline{28}.]  The LIGO Scientific Collaboration, the Virgo Collaboration, and the KAGRA Collaboration (R. Abbott, ..., \uline{Hiroki Takeda}, ..., et al.),
"\emph{All-sky search for continuous gravitational waves from isolated neutron stars using Advanced LIGO and Advanced Virgo O3 data}", Physical Review D, American Physical Society, {\bf 106}, 102008 (2022). (arXiv:2201.00697)\\

\item[\uline{29}.] The LIGO Scientific Collaboration, the Virgo Collaboration, and the KAGRA Collaboration (R. Abbott, ..., \uline{Hiroki Takeda}, ..., et al.) plus 28 radio astronomers and NICER science team members,
"\emph{Narrowband searches for continuous and long-duration transient gravitational waves from known pulsars in the LIGO-Virgo third observing run}", The Astrophysical Journal, IOP Publishing, {\bf 932}, 133 (2022). (arXiv:2112.10990)\\

\setcounter{enumi}{29}
\item The LIGO Scientific Collaboration, the Virgo Collaboration, and the KAGRA Collaboration (R. Abbott, ..., \uline{Hiroki Takeda}, ..., et al.), "\emph{Tests of General Relativity with GWTC-3}", Physical Review D, American Physical Society, (2021). (arXiv:2112.06861)\\

\item[\uline{31}.] The LIGO Scientific Collaboration, the Virgo Collaboration, and the KAGRA Collaboration (R. Abbott, ..., \uline{Hiroki Takeda}, ..., et al.), "\emph{All-sky search for gravitational wave emission from scalar boson clouds around spinning black holes in LIGO O3 data}", Physical Review D, American Physical Society, {\bf 105}, 102001 (2022). (arXiv:2111.15507)\\

\item[\uline{32}.] The LIGO Scientific Collaboration, the Virgo Collaboration, and the KAGRA Collaboration (R. Abbott, ..., \uline{Hiroki Takeda}, ..., et al.), "\emph{Searches for Gravitational Waves from Known Pulsars at Two Harmonics in the Second and Third LIGO-Virgo Observing Runs}", The Astrophysical Journal, IOP Publishing, {\bf935}, 1 (2021). (arXiv:2111.13106)\\

\setcounter{enumi}{32}
\item The LIGO Scientific Collaboration, the Virgo Collaboration, and the KAGRA Collaboration (R. Abbott, ..., \uline{Hiroki Takeda}, ..., et al.), "\emph{Constraints on the cosmic expansion history from the third LIGO-Virgo-KAGRA Gravitational-Wave Transient Catalog}", The Astrophysical Journal, IOP Publishing, (2021). (arXiv:2111.03604)\\

\item The LIGO Scientific Collaboration, the Virgo Collaboration, and the KAGRA Collaboration (R. Abbott, ..., \uline{Hiroki Takeda}, ..., et al.), "\emph{GWTC-3: Compact Binary Coalescences Observed by LIGO and Virgo During the Second Part of the Third Observing Run}", (2021). (arXiv:2111.03606)\\

\item[\uline{35}.] The LIGO Scientific Collaboration, the Virgo Collaboration, and the KAGRA Collaboration (R. Abbott, ..., \uline{Hiroki Takeda}, ..., et al.),
"\emph{Search for Gravitational Waves Associated with Gamma-Ray Bursts detected by Fermi and Swift during the O3b LIGO-Virgo Run}",
The Astrophysical Journal, IOP Publishing, {\bf 928}, 186 (2022).
(arXiv:2111.03608) \\

\item[\uline{36}.] The LIGO Scientific Collaboration, the Virgo Collaboration, and the KAGRA Collaboration (R. Abbott, ..., \uline{Hiroki Takeda}, ..., et al.), "\emph{The population of merging compact binaries inferred using gravitational waves through GWTC-3}", Physical Review X, American Physical Society, {\bf 13}, 011048 (2023). (arXiv:2111.03634)\\

\item[\uline{37}.] The LIGO Scientific Collaboration, the Virgo Collaboration, and the KAGRA Collaboration (R. Abbott, ..., \uline{Hiroki Takeda}, ..., et al.), "\emph{All-sky, all-frequency directional search for persistent gravitational waves from Advanced LIGO's and Advanced Virgo's first three observing runs}", Physical Review D, American Physical Society, {\bf 105}, 122002 (2022). (arXiv:2110.09834)\\

\item[\uline{38}.] The LIGO Scientific Collaboration, the Virgo Collaboration, the KAGRA Collaboration, plus D. Jeong and S. Shandera (R. Abbott, ..., \uline{Hiroki Takeda}, ..., et al.), "\emph{Search for subsolar-mass binaries in the first half of Advanced LIGO and Virgo's third observing run}", Physical Review Letters, American Physical Society, {\bf 129}, 061104 (2022). (arXiv:2109.12197)\\

\item[\uline{39}.] The LIGO Scientific Collaboration, the Virgo Collaboration, and the KAGRA Collaboration plus A. C. Albayati, D. Altamirano, P. Bult, D. Chakrabarty, M. Ng, P. S. Ray, A. Sanna, and T. E. Strohmayer (R. Abbott, ..., \uline{Hiroki Takeda}, ..., et al.), "\emph{Search for continuous gravitational waves from 20 accreting millisecond X-ray pulsars in O3 LIGO data}", Physical Review D, American Physical Society, {\bf 105}, 022002 (2021). (arXiv:2109.09255)\\

\item[\uline{40}.] The LIGO Scientific Collaboration, the Virgo Collaboration, and the KAGRA Collaboration (R. Abbott, ..., \uline{Hiroki Takeda}, ..., et al.), "\emph{All-sky search for long-duration gravitational-wave transients in the third Advanced LIGO observing run}", Physical Review D, American Physical Society, {\bf 104}, 102001 (2021). (arXiv:2107.13796)\\

\item[\uline{41}.] The LIGO Scientific Collaboration, the Virgo Collaboration, the KAGRA Collaboration (R. Abbott, ..., \uline{Hiroki Takeda}, ..., et al.), "\emph{All-sky search for short gravitational-wave bursts in the third Advanced LIGO and Advanced Virgo run}", Physical Review D, American Physical Society, {\bf 104}, 122004(2021). (arXiv:2107.03701)\\

\item[\uline{42}.] The LIGO Scientific Collaboration, the Virgo Collaboration, the KAGRA Collaboration (R. Abbott, ..., \uline{Hiroki Takeda}, ..., et al.), "\emph{All-sky Search for Continuous Gravitational Waves in the Early O3 LIGO Data}", Physical Review D, American Physical Society, {\bf 104}, 082004 (2021). (arXiv:2107.00600)\\

\item[\uline{43}.] The LIGO Scientific Collaboration, the Virgo Collaboration, the KAGRA Collaboration (R. Abbott, ..., \uline{Hiroki Takeda}, ..., et al.), "\emph{Observation of Gravitational Waves from Two Neutron Star\UTF{2013}Black Hole Coalescences}", The Astrophysical Journal Letters, IOP Publishing, {\bf 915}, L5 (2021). (arXiv:2012.12926)\\

\item[\uline{44}.] The LIGO Scientific Collaboration, the Virgo Collaboration, the KAGRA Collaboration (R. Abbott, ..., \uline{Hiroki Takeda}, ..., et al.), "\emph{Search for intermediate mass black hole binaries in the third observing run of Advanced LIGO and Advanced Virgo}", Astronomy \& Astrophysics, EDP Sciences, {\bf 659}, A84 (2021). (arXiv:2105.15120)\\

\item[\uline{45}.] The LIGO Scientific Collaboration, the Virgo Collaboration, the KAGRA Collaboration (R. Abbott, ..., \uline{Hiroki Takeda}, ..., et al.), "\emph{Constraints on dark photon dark matter using data from LIGO's and Virgo's third observing run}", Physical Review Letters, American Physical Society, {\bf 126}, 241102 (2021). (arXiv:2101.12248)\\

\item[\uline{46}.] The LIGO Scientific Collaboration, the Virgo Collaboration, and the KAGRA Collaboration (R. Abbott, ..., \uline{Hiroki Takeda}, ..., et al.), "\emph{Searches for continuous gravitational waves from young supernova remnants in the early third observing run of Advanced LIGO and Virgo}", The Astrophysical Journal, IOP Publishing, {\bf 921}, 80 (2021). (arXiv:2105.11641)\\

\item[\uline{47}.] The LIGO Scientific Collaboration, the Virgo Collaboration, and the KAGRA Collaboration plus D. Antonopoulou, Z. Arzoumanian, T. Enoto, C. M. Espinoza, and S. Guillot (R. Abbott, ..., \uline{Hiroki Takeda}, ..., et al.), "\emph{Constraints from LIGO O3 data on gravitational-wave emission due to r-modes in the glitching pulsar PSR J0537-6910 }", The Astrophysical Journal, IOP Publishing, {\bf 922}, 71 (2021). (arXiv:2104.14417)\\

\item[\uline{48}.] The LIGO Scientific Collaboration, the Virgo Collaboration, the KAGRA Collaboration (R. Abbott, ..., \uline{Hiroki Takeda}, ..., et al.), "\emph{Search for anisotropic gravitational-wave backgrounds using data from Advanced LIGO's and Advanced Virgo's first three observing runs }", Physical Review D, American Physical Society, {\bf 104}, 022005 (2021). (arXiv:2103.08520)\\

\item[\uline{49}.] The LIGO Scientific Collaboration, the Virgo Collaboration, the KAGRA Collaboration (R. Abbott, ..., \uline{Hiroki Takeda}, ..., et al.), "\emph{Constraints on Cosmic Strings Using Data from the Third Advanced LIGO–Virgo Observing Run}", Physical Review Letters, American Physical Society, {\bf 126}, 241102 (2021). (arXiv:2101.12248)\\

\item[\uline{50}.] The LIGO Scientific Collaboration, the Virgo Collaboration, the KAGRA Collaboration (R. Abbott, ..., \uline{Hiroki Takeda}, ..., et al.), "\emph{Upper Limits on the Isotropic Gravitational-Wave Background from Advanced LIGO's and Advanced Virgo's Third Observing Run}", Physical Review D, American Physical Society, {\bf 104}, 022004 (2021). (arXiv:2101.12130)\\

\item[\uline{51}.] The LIGO Scientific Collaboration, the Virgo Collaboration, the KAGRA Collaboration (R. Abbott, ..., \uline{Hiroki Takeda}, ..., et al.), "\emph{Diving below the Spin-down Limit: Constraints on Gravitational Waves from the Energetic Young Pulsar PSR J0537-6910}", The Astrophysical Journal Letters, IOP Publishing, {\bf 913}, L27 (2021). (arXiv:2012.12926)\\

\item[\uline{52}.] KAGRA Collaboration (T. Akutsu, ..., \uline{Hiroki Takeda}, ..., et al.),
"\emph{Vibration isolation systems for the beam splitter and signal recycling mirrors of the KAGRA gravitational wave detector}",
Classical and Quantum Gravity, IOP Publishing, {\bf 38}, 065011 (2021).\\

\item[\uline{53}.] KAGRA Collaboration (T. Akutsu, ..., \uline{Hiroki Takeda}, ..., et al.), "\emph{Overview of KAGRA: Calibration, detector characterization, physical environmental monitors, and the geophysics interferometer}", Progress of Theoretical and Experimental Physics, Physical Society of Japan, {\bf 2020}, ptab018 (2020). (arXiv:2009.09305)\\

\item[\uline{54}.] KAGRA Collaboration (T. Akutsu, ..., \uline{Hiroki Takeda}, ..., et al.), "\emph{Overview of KAGRA: KAGRA science}", Progress of Theoretical and Experimental Physics, Physical Society of Japan, {\bf 2020}, ptaa120 (2020). (arXiv:2008.02921)\\

\item[\uline{55}.] KAGRA Collaboration (T. Akutsu, ..., \uline{Hiroki Takeda}, ..., et al.), "\emph{Overview of KAGRA: Detector design and construction history}", Progress of Theoretical and Experimental Physics, Physical Society of Japan, {\bf 2020}, ptaa125 (2020). (arXiv:2005.05574)\\

\item[\uline{56}.] KAGRA Collaboration, LIGO Scientific Collaboration and Virgo Collaboration (B.P. Abbott, ..., \uline{Hiroki Takeda}, ..., et al.), "\emph{Prospects for observing and localizing gravitational-wave transients with Advanced LIGO, Advanced Virgo and KAGRA}", Living Reviews in Relativity, Springer International Publishing, {\bf 23}, 3 (2020).\\

\item[\uline{57}.] Enrico Barausse, ..., \uline{Hiroki Takeda}, ..., et al., "\emph{Prospects for fundamental physics with LISA}", General Relativity and Gravitation, International Society on General Relativity and Gravitation, {\bf 52}, 81 (2020). (arXiv:2001.09793)\\

\item[\uline{58}.] KAGRA Collaboration (T. Akutsu, ..., \uline{Hiroki Takeda}, ..., et al.), "\emph{Application of the independent component analysis to the iKAGRA data}", Progress of Theoretical and Experimental Physics, Physical Society of Japan, {\bf 2020}, 053F01 (2020). (arXiv:1908.03013)\\

\item[\uline{59}.] KAGRA Collaboration (Tomotada Akutsu, ..., \uline{Hiroki Takeda}, ..., et al.), "\emph{An arm length stabilization system for KAGRA and future gravitational-wave detectors}", Classical and Quantum Gravity, IOP Publishing, {\bf 37}, 035004 (2020). (arXiv:1910.00955)\\

\item[\uline{60}.] KAGRA Collaboration (Tomotada Akutsu, ..., \uline{Hiroki Takeda}, ..., et al.), "\emph{First cryogenic test operation of underground km-scale gravitational-wave observatory KAGRA}", Classical and Quantum Gravity, IOP Publishing, {\bf 36}, 165008 (2019). (arXiv:1901.03569)\\

\item[\uline{61}.] KAGRA Collaboration (Y. Akiyama, ..., \uline{Hiroki Takeda}, ..., et al.), "\emph{Vibration isolation system with a compact damping system for power recycling mirrors of KAGRA}", Classical and Quantum Gravity, IOP Publishing, {\bf 36}, 095015 (2019). (arXiv:1901.03053)\\

\item[\uline{62}.] KAGRA Collaboration (Tomotada Akutsu, ..., \uline{Hiroki Takeda}, ..., et al.), "\emph{KAGRA: 2.5 Generation Interferometric Gravitational Wave Detector}", Nature Astronomy, Nature Publishing Group, {\bf 3}, 35 (2019). (arXiv:1811.08079)\\

\item[\uline{63}.] KAGRA Collaboration (Tomotada Akutsu, ..., \uline{Hiroki Takeda}, ..., et al.), "\emph{Construction of KAGRA: an Underground Gravitational Wave Observatory}", Progress of Theoretical and Experimental Physics, Physical Society of Japan, {\bf 2018}, 1 (2018). (arXiv: 1712.00148)

\end{enumerate}


\section*{Presentations}

\subsection*{Invited talks}
\begin{enumerate}
\item \uline{Hiroki Takeda}, "\emph{Probing scalar polarizations in gravitational waves}", GW research exchange meeting, online, June 2023.\\

\item \uline{Hiroki Takeda}, "\emph{Scalar polarization window in gravitational-wave signals from compact binary coalescences}",
Astrophysical Group Seminar, Waseda University, Tokyo, Japan, May 2023.\\

\item \uline{Hiroki Takeda}, "\emph{Polarization tests by DECIGO}", 21st DECIGO workshop, remote style, Dec. 2022.\\

\item \uline{Hiroki Takeda}, "\emph{Consistent search for polarization modes of gravitational waves from generation to detection}", Observational Cosmology Summer Workshop in 2022,  Hamamatsu, Japan, Aug. 2022.\\

\item \uline{Hiroki Takeda}, "\emph{Polarization test of gravitational waves from compact binary coalescences}", Seminar at Institute of Particle Physics and Astrophysics, School of Physics, Huazhong University of Science and Technology, Wuhan, China, Nov. 2018.\\
\end{enumerate}


\subsection*{International conference}
\subsubsection*{Oral presentations}
\begin{enumerate}
\setcounter{enumi}{5}
\item \uline{Hiroki Takeda}, Yusuke Manita, Hidetoshi Omiya and Takahiro Tanaka, "\emph{Scalar polarization window in gravitational-wave signals}", 15th Edoardo Amaldi Conference on Gravitational Waves, online, Jul. 2023.\\

\item \uline{Hiroki Takeda}, Soichiro Morisaki, and Atsushi Nishizawa “\emph{Search for mixture of scalar-tensor polarizations of gravitational waves}”, 23rd International Conference on General Relativity and Gravitation, Beijing, China, hybrid, Jul. 2022.\\

\item \uline{Hiroki Takeda}, Soichiro Morisaki, and Atsushi Nishizawa,
"\emph{Testing gravity theory with gravitational wave polarizations}",
Innovative Area "Gravitational Wave Physics and Astronomy: Genesis" Group A Winter Camp,Kyoto, Japan, Jan. 2021.\\

\item \uline{Hiroki Takeda}, Soichiro Morisaki, and Atsushi Nishizawa “\emph{Scalar-tensor mixed polarization search for gravitational waves from compact binary coalescences}”, Gravitational Wave Physics and Astronomy Workshop 2021, Hannover, Germany, Dec. 2021.\\

\item \uline{Hiroki Takeda}, Atsushi Nishizawa, and Soichiro Morisaki, "\emph{Polarization tests of GW170814 and GW170817 using waveforms consistent with alternative theories of gravity}", 7th KAGRA International Workshop, In person for Taiwan people, in remote for others, Dec. 2020.\\

\item \uline{Hiroki Takeda}, Tomoya Kinugawa, and Hiroya Yamaguchi, "\emph{Ability of DECIGO to constrain the Type Ia supernova progenitor system}", Gravitational Wave Physics and Astronomy Workshop 2019, Tokyo, Japan, Oct. 2019.\\

\item \uline{Hiroki Takeda}, Atsushi Nishizawa, Yuta Michimura, Koji Nagano, Kentaro Komori, Masaki Ando, and Kazuhiro Hayama, "\emph{Probing nontensorial polarization of inspiral gravitational waves with the third-generation detectors}", The 28th Workshop on General Relativity and Gravitation in Japan – JGRG28, Tokyo, Japan, Nov. 2018.\\

\item \uline{Hiroki Takeda}, Atsushi Nishizawa, Yuta Michimura, Koji Nagano, Kentaro Komori, Masaki Ando, and Kazuhiro Hayama, "\emph{Polarization test of gravitational waves from compact binary coalescences}", The 15th Marcel Grossmann Meeting, Rome, Italy, Jul. 2018.\\

\item \uline{Hiroki Takeda}, Atsushi Nishizawa, Yuta Michimura, Koji Nagano, Kentaro Komori, Masaki Ando, and Kazuhiro Hayama, “\emph{Parameter estimation with inspiral waveforms of compact binary coalescences including nontensorial gravitational waves polarizations}”, 19th KAGRA face-to-face meeting, Osaka, Japan, May. 2018.
\end{enumerate}

\subsubsection*{Poster presentations}

\begin{enumerate}
\setcounter{enumi}{14}
\item \uline{Hiroki Takeda}, Yusuke Manita, Hidetoshi Omiya, Takahiro Tanaka, “\emph{Scalar gravitational wave and fifth force}”, 28th KAGRA Face to Face Meeting, in remote style, Dec. 2022.\\

\item \uline{Hiroki Takeda}, Atsushi Nishizawa, and Soichiro Morisaki, “\emph{Search for scalar-tensor mixed polarization of gravitational waves}”, 14th Edoardo Amaldi Conference on Gravitational Waves, in remote style, Jul. 2021.\\

\item \uline{Hiroki Takeda}, Atsushi Nishizawa, and Soichiro Morisaki, “\emph{Tests of alternative theories of gravity through gravitational-wave polarization modes}”, 26th Face to Face Meeting, in remote style, Dec. 2020.\\

\item \uline{Hiroki Takeda}, Atsushi Nishizawa, Yuta Michimura, Koji Nagano, Kentaro Komori, Masaki Ando, and Kazuhiro Hayama, “\emph{Prospects for gravitational-wave polarization test from compact binary coalescences with next-generation detectors}”, 22nd International Conference on General Relativity and Gravitation, 13th Edoardo Amaldi Conference on Gravitational Waves, Valencia, Spain, Jul. 2019.\\

\end{enumerate}


\section*{Outreach}

\begin{enumerate}

\item \uline{Hiroki Takeda}, KagaQ "Science on a Moonlit Night: Verification of General Relativity and Gravitational Wave Observation," Lecture, November 27, 2023.\\

\item \uline{Hiroki Takeda}, "Researchers' Twitter Strategy" at Kyoto University's Shirahae Center Symposium, Guest Speaker, November 20, 2023.\\

\item \uline{Hiroki Takeda}, Asahi Shimbun Morning Edition "Jo-Ha-Kyu," Interview Article, October 16, 2023.\\

\item \uline{Hiroki Takeda}, Asahi Shimbun Evening Edition "Good Gourmet," Interview Article, October 12, 2023.\\

\item \uline{Hiroki Takeda}, "Doctor Idol Project Final Audition" at Akihabara UDX Event, Concept and Planning, October 8, 2023.\\

\item \uline{Hiroki Takeda}, Mainichi Shimbun Morning Edition "Mizu Setsu," Interview Article, October 4, 2023.\\

\item \uline{Hiroki Takeda}, "President Family Autumn Edition" by President Company, Interview Article, September 5, 2023.\\

\item \uline{Hiroki Takeda}, Kyodo News Agency "47NEWS," Interview Article, May 13, 2023.\\

\item \uline{Hiroki Takeda}, Yomiuri Shimbun Morning Edition "The Gate of News," Interview Article, April 13, 2023.\\

\item \uline{Hiroki Takeda}, TBS Radio "Ashita no College," Guest Live Appearance, February 11, 2023.\\

\item \uline{Hiroki Takeda}, "Exploring the Mysteries of the Universe with Gravity" for High School Girls at Kyoto University Faculty of Science - Universe \& Biology, Lecture, March 5, 2023.\\

\item \uline{Hiroki Takeda}, "Exploring the Universe through Gravitational Waves - What Awaits Beyond Days of Testing" at Minoh Free Academy High School, Lecture, August 4, 2022.\\

\item \uline{Hiroki Takeda}, NHK Kyoto "Kyoto One Day," Guest Live Appearance, July 27, 2022.\\

\item \uline{Hiroki Takeda}, Mainichi Shimbun Evening Edition "Doctor Idol - Yūrakuchō," Interview Article, June 20, 2022.\\

\item \uline{Hiroki Takeda}, YouTube Channel “4-Panel Universe,” Total Views 278,000, since February 2022.\\
\end{enumerate}

\end{document}
