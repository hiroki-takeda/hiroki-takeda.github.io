%%%%%%%%% JPS abstruct %%%%%%%%%%%%%%%%%%%%%%%%%%%%%%%%%%%%%%%%%%
\documentclass[uplatex, 11pt]{jsarticle}
\special{papersize=210truemm,297truemm}

%%%%%%%%% packages %%%%%%%%%%%%%%%%%%%%%%%%%%%%%%%%%%%%%%%%%%%%%%
\usepackage{graphicx} % Include figure files
\usepackage{geometry} %余白設定用package
\geometry{inner=2cm, outer=2cm, top=2cm, bottom=2cm}
\usepackage[dvipdfmx,bookmarks=true,bookmarksnumbered=true,colorlinks=true,linkcolor=blue,citecolor=blue,filecolor=blue,urlcolor=blue]{hyperref} %hyperref packageを使って目次にリンク(しおり)を貼る。
\usepackage[dvipdfmx]{pxjahyper} %日本語文字化け防止
\usepackage[normalem]{ulem} %下線引きたい

\usepackage{multicol} %部分的に二段組
\setlength{\columnsep}{-3.5cm} %二段組の幅調整

\usepackage{otf} % 名前用フォント

\pagestyle{empty}

%enumerateで数字に丸印つける。
\newcommand{\ctext}[1]{\ooalign{
\hfil\resizebox{\width}{\height}{#1}\hfil
\crcr
\raise-.1mm\hbox{\Large$\bigcirc$}}}

\newcommand{\comicneue}{\selectfont}
\DeclareTextFontCommand{\textcomicneue}{\comicneue}

% New environment for the long list
\newenvironment{cvlist}{%
	\begin{tabular*}{\textwidth}{@{\extracolsep{\fill}}ll}
}{%
	\end{tabular*}
}

%\cvitem{<dates>}{<title>}{<location>}{<description>}
\newcommand{\cvitem}[4]{%
#1&\parbox[t]{0.83\textwidth}{%
	\normalfont{#2}%
	\hfill%
	{\footnotesize#3}\\%
	#4\vspace{\parsep}%
	}\\
}

%%%%%%%%% header %%%%%%%%%%%%%%%%%%%%%%%%%%%%%%%%%%%%%%%%%%%%%%%%
\begin{document}

\begin{center}
\textcomicneue{\huge\textgt{履歴書}}\\[14pt]

\textcomicneue{\Huge\textgt{武田 紘樹}}\\[5pt]

\end{center}
\date{\today}

\vspace{20pt}
%%%%%%%%%%%%%%%%%%%%%%%%%%%%%%%%%%%%%%%%%%%%%%%%%%%%%%%%%%%

\begin{multicols}{2}
\noindent 
\textbf{住所:} 
〒606-8502 \\
\noindent 
京都市左京区北白川追分町\par
\noindent 
京都大学理学部物理学第二教室\\
\noindent 
天体核研究室\\
\columnbreak

\noindent 
\textbf{電話番号:} +81-75-753-3876\\
\noindent 
\textbf{E-mail:} takeda@tap.scphys.kyoto-u.ac.jp\\
\noindent 
\textbf{Webサイト:} https://hiroki-takeda.github.io/index.html\\
\noindent 
\textbf{生年月日:} 1993年10月5日
\end{multicols}

%%%%%%%%%%%%%%%%%%%%%%%%%%%%%%%%%%%%%%%%%%%%%%%%%%%%%%%%%%%
\section*{学歴}
\noindent
\begin{cvlist} % Environment for a list with descriptions
	%\cvitem{<dates>}{<title>}{<location>}{<description>}
\cvitem{2018-2021}{東京大学大学院理学系研究科物理学専攻博士課程,}{}{\emph{Tests of Alternative Theories of Gravity through Gravitational-Wave Polarizations,} \\取得年月:  2021年3月.}\\
\cvitem{2016-2018}{東京大学大学院理学系研究科物理学専攻修士課程,}{}{\emph{Development of Monolithic Optical System for Lorentz Invariance Test,}\\ 取得年月:  2018年3月.}\\
\cvitem{2012-2016}{横浜国立大学理工学部数物・電子情報系学科物理工学EP,}{}{\emph{Proposal of Rayleigh Scattering Length Measurement in Liquid Xenon,}\\ 取得年月:  2016年3月.}
\end{cvlist}

\section*{職歴}
\noindent
\begin{cvlist} % Environment for a list with descriptions
	%\cvitem{<dates>}{<title>}{<location>}{<description>}
	\cvitem{2023-現在}{特定助教,}{}{京都大学白眉センター.}\\
	\cvitem{2023-現在}{連携助教,}{}{京都大学大学院理学研究科.}\\	
	\cvitem{2021-2023}{日本学術振興会特別研究員(PD),}{}{京都大学大学院理学研究科
物理学・宇宙物理学専攻 物理第二分野.}\\
	\cvitem{2018-2021}{日本学術振興会特別研究員(DC1),}{}{東京大学大学院理学系研究科物理学専攻.}\\
	\cvitem{2016-2021}{Advanced Leading Graduate Course for Photon Science 生,}{}{東京大学大学院理学系研究科物理学専攻.}
\end{cvlist}

\section*{受賞歴}
\noindent
\begin{cvlist} % Environment for a list with descriptions
	%\cvitem{<dates>}{<title>}{<location>}{<description>}
	\cvitem{2021}{GWIC-Braccini Prize Honorable Mention}{}{Gravitational Wave International Committee.}\\
	\cvitem{2021}{第76回年次大会(2021年)日本物理学会学生優秀発表賞.}{}{}
	\cvitem{2015}{横浜国立大学理工学部数物・電子情報系学科成績優秀賞.}{}{}
	\cvitem{2015}{横浜国立大学横浜物理工学会同窓会優秀賞.}{}{}
	\cvitem{2014}{横浜国立大学理工学部数物・電子情報系学科物理工学EP I実習優秀ポスター賞.}{}{}
\end{cvlist}

\section*{助成金}
\noindent
\begin{cvlist} % Environment for a list with descriptions
	%\cvitem{<dates>}{<title>}{<location>}{<description>}
	\cvitem{2024-2029}{京都大学白眉プロジェクト 研究代表者,}{}{研究題目: コンパクト連星合体からの重力波の偏極モード探査による極限環境での重力理論検証,\\ 総額1432万円.}\\
	\cvitem{2022-2026}{日本学術振興会 若手研究 研究代表者,}{}{研究題目: 重力波伝播過程の偏極モード探査による宇宙論的距離スケールでの重力理論の検証,\\ 総額360万円.}\\
	\cvitem{2021-2024}{日本学術振興会特別研究員(PD)  特別研究員奨励費 研究代表者,}{}{研究題目: コンパクト連星合体からの重力波偏極モード探査による強重力場での重力理論検証,\\ 総額364万円.}\\
	\cvitem{2018-2021}{日本学術振興会特別研究員(DC1)  特別研究員奨励費 研究代表者,}{}{研究題目: 光リング共振器を用いた光子のローレンツ不変性検証,\\ 総額280万円.}
\end{cvlist}

\section*{教育歴}
\noindent
\begin{cvlist} % Environment for a list with descriptions
	%\cvitem{<dates>}{<title>}{<location>}{<description>}
	\cvitem{2025-}{京都大学 理学部物理学科 4年生 「物理科学課題研究P5」卒業研究に相当.}{}{}
	\cvitem{2025-}{龍谷大学 3-4年生 非常勤講師「複雑系の科学」.}{}{}
	\cvitem{2022-2023}{舞鶴工業高等専門学校 機械工学科 4年生 非常勤講師「物理学III(力学)」.}{}{}
	\cvitem{2018-2019}{東京大学 理学部 物理学科 学部3年 「学生実験II(ブラウン運動)」TA.}{}{}
	\cvitem{2012-2016}{武蔵ゼミナール 進学教室スクール8教室 学習塾講師.}{}{}
\end{cvlist}

\section*{委員等}
\noindent
\begin{cvlist} % Environment for a list with descriptions
	%\cvitem{<dates>}{<title>}{<location>}{<description>}
	\cvitem{2026年1月}{国際会議「JGRG34」 Local Organizing Committeeメンバー.}{}{}
	\cvitem{2025年10月}{研究会「ACG2025」 オーガナイザー(chair).}{}{}
	\cvitem{2025年}{白眉センター 部局情報セキュリティ技術責任者.}{}{}
	\cvitem{2025年3月}{研究会「Gravitational waves Related workshop in Western Japan」 オーガナイザー(chair).}{}{}
	\cvitem{2024年}{白眉センター HPワーキンググループ.}{}{}
	\cvitem{2024年}{白眉センター 14期PRグループ代表.}{}{}
	\cvitem{2020年}{KAGRA Scientific Congress 学生代表.}{}{}
\end{cvlist}

\section*{所属団体}
\noindent
\begin{cvlist} % Environment for a list with descriptions
	%\cvitem{<dates>}{<title>}{<location>}{<description>}
	\cvitem{2019-現在}{日本天文学会 会員.}{}{}
	\cvitem{2018-現在}{LISA Consortium メンバー.}{}{}
	\cvitem{2017-現在}{KAGRA Collaboration.}{}{}
	\cvitem{2016-現在}{日本物理学会 会員.}{}{}
\end{cvlist}

\section*{指標}
%BEGIN_METRICS%
%END_METRICS%

\section*{出版物}
%BEGIN_PUBLICATIONS%
%END_PUBLICATIONS%

\section*{発表}
%BEGIN_PRESENTATIONS%
%END_PRESENTATIONS%

\section*{アウトリーチ活動}
%BEGIN_ACTIVITIES%
%END_ACTIVITIES%


\end{document}